\begin{minted}[frame=lines, bgcolor=bg, linenos]{go}
package main

import (
    "fmt"
    "os"
    "time"

    "k8s.io/client-go/informers"
    "k8s.io/client-go/kubernetes"
    "k8s.io/client-go/tools/cache"
    "k8s.io/client-go/tools/clientcmd"
)

func main() {
    // Laden der Kubernetes-Konfiguration (innerhalb oder außerhalb des Clusters)
    config, err := clientcmd.BuildConfigFromFlags("", clientcmd.RecommendedHomeFile)
    if err != nil {
        fmt.Fprintf(os.Stderr, "Error building kubeconfig: %s\n", err.Error())
        os.Exit(1)
    }
    // Erstellen des Clientsets
    clientset, err := kubernetes.NewForConfig(config)
    if err != nil {
        fmt.Fprintf(os.Stderr, "Error creating clientset: %s\n", err.Error())
        os.Exit(1)
    }
    // Erstellen eines Factory für SharedInformers
    factory := informers.NewSharedInformerFactory(clientset, 10*time.Minute)
    // Informer für Pods erstellen
    podInformer := factory.Core().V1().Pods().Informer()
    stopCh := make(chan struct{})
    defer close(stopCh)
    // Event-Handler für den Pod-Informer hinzufügen
    podInformer.AddEventHandler(cache.ResourceEventHandlerFuncs{
        AddFunc: func(obj interface{}) {
            fmt.Println("Pod added:", obj)
        },
        UpdateFunc: func(oldObj, newObj interface{}) {
            fmt.Println("Pod updated:", newObj)
        },
        DeleteFunc: func(obj interface{}) {
            fmt.Println("Pod deleted:", obj)
        },
    })
    // Informer starten
    go podInformer.Run(stopCh)
    // Warten auf Synchronisierung der Caches
    if !cache.WaitForCacheSync(stopCh, podInformer.HasSynced) {
        fmt.Fprintf(os.Stderr, "Error waiting for cache sync\n")
        os.Exit(1)
    }
    fmt.Println("Controller running")
    select {}
}

\end{minted}