\chapter{Workloads und Ressourcen}

\section{DaemonSets}
Stellen sicher, dass eine Kopie eines Pods auf jeder Node (oder einer ausgewählten Gruppe von Nodes) im Cluster läuft. \\

\noindent
\begin{tabular}{|l|l|}
\hline
\textbf{Befehl} & \textbf{Beschreibung} \\
\hline
\texttt{kubectl get daemonsets} & Alle DaemonSets auflisten \\
\texttt{kubectl describe daemonset <daemonset-name>} & Details zu einem DaemonSet anzeigen \\
\texttt{kubectl create -f <filename>} & DaemonSet aus einer Datei erstellen\\
\texttt{kubectl apply -f <filename>} & DaemonSet aus einer Datei erstellen, oder aktualisieren\\
\texttt{kubectl delete daemonset <daemonset-name>} & Ein DaemonSet löschen \\
\hline
\end{tabular}
\subsubsection{YAML-Datei für ein DaemonSet}
\begin{minted}[frame=lines, bgcolor=bg]{yaml}
apiVersion: apps/v1
kind: DaemonSet
metadata:
  name: nginx-daemonset
  labels:
    app: nginx
spec:
  selector:
    matchLabels:
      app: nginx
  template:
    metadata:
      labels:
        app: nginx
    spec:
      containers:
        - name: nginx
          image: nginx:latest
          ports:
            - containerPort: 80
\end{minted}

\subsection{Anwendungsfälle für DaemonSets}
\begin{itemize}
    \item Protokollierung: Ein Protokollierungs-Agent muss auf jeder Node laufen, um die Logs von Anwendungen zu sammeln.
    \item Überwachung: Überwachungs-Agents sammeln Metriken von Nodes und Pods und senden sie an ein zentrales Überwachungssystem.
    \item Netzwerk-Dienste: Bereitstellung von Netzwerkdiensten wie DNS oder CNI-Plugins, die auf jeder Node laufen müssen.
\end{itemize}
\newpage
\subsection{Best Practices für DaemonSets}
\begin{itemize}
    \item Ressourcenanforderungen und limits setzen, um die Stabilität des Clusters zu gewährleisten.
    \item Node-Selector verwenden, wenn DaemonSets nur auf bestimmten Nodes ausgeführt werden sollen
    \item PodAntiAffinity verwenden, um sicherzustellen dass DaemonSet-Pods nicht auf denselben Nodes wie andere kritische Pods laufen.
    \item Rolling Updates verwenden, um eine reibungslose Aktualisierung der DaemonSet-Pods ohne Ausfallzeiten zu gewährleisten.
\end{itemize}
\subsection{Weitere nützliche Befehle für DaemonSets}
\begin{tabular}{|p{0.53\textwidth}|p{0.47\textwidth}|}
\hline
\textbf{Befehl} & \textbf{Beschreibung} \\
\hline
\texttt{kubectl rollout status daemonset/<daemonset-name>} & Den Rollout-Status eines DaemonSets anzeigen \\
\texttt{kubectl edit daemonset <daemonset-name>} & Ein DaemonSet direkt im Editor bearbeiten \\
\texttt{kubectl scale --replicas=<number> daemonset <daemonset-name>} & Die Anzahl der Pods in einem DaemonSet ändern \\
\texttt{kubectl get daemonset <daemonset-name> -o yaml} & Die vollständige YAML-Konfiguration eines DaemonSets anzeigen \\
\texttt{kubectl patch daemonset <daemonset-name> -p <patch-data>} & Einen Patch auf ein DaemonSet anwenden \\
\hline
\end{tabular}

\subsection{Nützliche Links und Ressourcen}

Kubernetes DaemonSet Dokumentation:\\
\url{https://kubernetes.io/docs/concepts/workloads/controllers/daemonset/}\\
Kubernetes Tasks für DaemonSets:\\
\url{https://kubernetes.io/docs/tasks/manage-daemon/}\\

\newpage
\section{StatefulSets}
\label{sec:statefulsets}

Verwalten den Zustand und die Identität von Pods, insbesondere für zustandsbehaftete Anwendungen. \\

\noindent
\begin{tabular}{
|l|l|}
\hline
\textbf{Befehl} & \textbf{Beschreibung} \\
\hline
\texttt{kubectl get statefulsets} & Alle StatefulSets auflisten \\
\texttt{kubectl describe statefulset <statefulset-name>} & Details zu einem StatefulSet anzeigen \\
\texttt{kubectl apply -f <dateiname>} & StatefulSet aus Datei erstellen\\
\texttt{kubectl delete statefulset <statefulset-name>} & Ein StatefulSet löschen \\
\hline
\end{tabular}

\subsubsection{Beispielkonfiguration für ein StatefulSet}
\begin{minted}[frame=lines, bgcolor=bg]{yaml}
apiVersion: apps/v1
kind: StatefulSet
metadata:
  name: web
spec:
  selector:
    matchLabels:
      app: nginx
  serviceName: nginx
  replicas: 3
  template:
    metadata:
      labels:
        app: nginx
    spec:
      containers:
        - name: nginx
          image: nginx:latest
          ports:
            - containerPort: 80
          volumeMounts:
            - name: www
              mountPath: /usr/share/nginx/html
  volumeClaimTemplates:
    - metadata:
        name: www
      spec:
        accessModes:
          - ReadWriteOnce
        resources:
          requests:
            storage: 1Gi
\end{minted}

\newpage
\subsection{Anwendungsfälle für StatefulSets}
\begin{itemize}
    \item Datenbanken: StatefulSets bieten stabile Netzwerk-IDs und persistente Speicher, die für Datenbanken erforderlich sind.
    \item Verteilte Dateisysteme: Verteilte Dateisysteme wie HDFS (siehe \ref{subsec:verteilte-dateisysteme}) benötigen stabile IDs und persistente Speicher.
    \item Clustered Anwendungen: Anwendungen, die auf Clustering angewiesen sind, benötigen stabile Netzwerk-IDs.
\end{itemize}
\subsection{Best Practices für StatefulSets}
\begin{itemize}
    \item Headless Services (siehe \ref{subsec:headless-service}) verwenden, um DNS-Namen für individuelle Pods zu ermöglichen
    \item Ressourcenanforderungen und -limits setzen, um die Stabilität des Clusters zu gewährleisten.
    \item PersistentVolumes verwenden, um sicherzustellen, dass der Speicher nach Pod-Neustarts erhalten bleibt.
    \item PodManagementPolicy verwenden, um die Verwaltung der Pods feiner zu steuern.
\end{itemize}
\subsection{Weitere nützliche Befehle für StatefulSets}
\begin{tabular}{|p{0.7\textwidth}|p{0.3\textwidth}|}
\hline
\textbf{Befehl} & \textbf{Beschreibung} \\
\hline
\texttt{kubectl rollout status statefulset/<statefulset-name>} & Rollout-Status anzeigen \\
\texttt{kubectl edit statefulset <statefulset-name>} & StatefulSet bearbeiten \\
\texttt{kubectl scale --replicas=<number> statefulset <statefulset-name>} & Anzahl der Pods ändern \\
\texttt{kubectl get statefulset <statefulset-name> -o yaml} & Konfiguration anzeigen \\
\texttt{kubectl patch statefulset <statefulset-name> -p <patch-data>} & Patch anwenden \\
\hline
\end{tabular}

\subsection{Nützliche Links und Ressourcen}
Kubernetes StatefulSet Dokumentation:\\
\url{https://kubernetes.io/docs/concepts/workloads/controllers/statefulset/}\\
Kubernetes Task: Skalieren von StatefulSets:\\
\url{https://kubernetes.io/docs/tasks/run-application/scale-stateful-set/}\\
Kubernetes Tutorial: Basic StatefulSet:\\
\url{https://kubernetes.io/docs/tutorials/stateful-application/basic-stateful-set/}\\



\section{ReplicaSets}
ReplicaSets stellen sicher, dass eine definierte Anzahl von Pod-Replikaten zu jeder Zeit läuft. Sie sind der Nachfolger von ReplicationControllers und bieten zusätzliche Selektionsmöglichkeiten. \\

\noindent
\begin{tabular}{|p{0.7\textwidth}|p{0.3\textwidth}|}
\hline
\textbf{Befehl} & \textbf{Beschreibung} \\
\hline
\texttt{kubectl get replicasets} & ReplicaSets auflisten \\
\texttt{kubectl describe replicaset <replicaset-name>} & Details anzeigen \\
\texttt{kubectl create -f <replicaset.yaml>} & ReplicaSet aus Datei erstellen \\
\texttt{kubectl delete replicaset <replicaset-name>} & ReplicaSet löschen \\
\texttt{kubectl scale replicaset <replicaset-name> --replicas=<number>} & Anzahl der Pods skalieren \\
\texttt{kubectl get pods --selector=<label>=<value>} & Zugeordnete Pods anzeigen \\
\texttt{kubectl edit replicaset <replicaset-name>} & ReplicaSet bearbeiten \\
\texttt{kubectl patch replicaset <replicaset-name> -p <patch-data>} & ReplicaSet patchen \\
\hline
\end{tabular}
\newpage
\subsubsection{YAML-Datei für ein ReplicaSet}
\begin{minted}[frame=lines, bgcolor=bg]{yaml}
apiVersion: apps/v1
kind: ReplicaSet
metadata:
  name: nginx-replicaset
spec:
  replicas: 3
  selector:
    matchLabels:
      app: nginx
  template:
    metadata:
      labels:
        app: nginx
    spec:
      containers:
        - name: nginx
          image: nginx:latest
          ports:
            - containerPort: 80

\end{minted}

\subsection{Anwendungsfälle für ReplicaSets}
\begin{itemize}
    \item Sicherstellen der Hochverfügbarkeit: ReplicaSets stellen sicher, dass eine bestimmte Anzahl von Pod-Replikaten immer läuft, um Ausfallzeiten zu minimieren.
    \item Lastverteilung: Durch das Erstellen mehrerer Replikate kann die Last auf mehrere Pods verteilt werden.
    \item Skalierung: Einfaches horizontales Skalieren von Anwendungen durch Ändern der Anzahl der Replikate.
\end{itemize}

\subsection{Best Practices für ReplicaSets}
\begin{itemize}
    \item Labels und Selektoren verwenden, um sicherzustellen, dass nur die gewünschten Pods vom ReplicaSet verwaltet werden.
    \item Ressourcenanforderungen und -limits setzen, um die Stabilität und Effizienz des Clusters zu gewährleisten.
    \item ReplicaSets regelmäßig überwachen, um sicherzustellen, dass sie wie erwartet funktionieren und die gewünschte Anzahl von Pods läuft.
    \item den Befehl \enquote{\texttt{kubectl rollout}} verwenden, um Änderungen und Updates an ReplicaSets zu überwachen und zu kontrollieren.
\end{itemize}

\subsection{Weitere nützliche Befehle für ReplicaSets}
\begin{tabular}{|p{0.53\textwidth}|p{0.47\textwidth}|}
\hline
\textbf{Befehl} & \textbf{Beschreibung} \\
\hline
\texttt{kubectl rollout status replicaset/<replicaset-name>} & Den Rollout-Status eines ReplicaSets anzeigen \\
\texttt{kubectl rollout history replicaset/<replicaset-name>} & Die Rollout-Historie eines ReplicaSets anzeigen \\
\texttt{kubectl set image replicaset/<replicaset-name> <container-name>=<new-image>} & Das Image eines Containers in einem ReplicaSet aktualisieren \\
\texttt{kubectl get rs -o wide} & Detaillierte Informationen zu allen ReplicaSets anzeigen \\
\texttt{kubectl get pods -l <label>=<value>} & Pods anzeigen, die einem bestimmten ReplicaSet zugeordnet sind \\
\texttt{kubectl delete pod --force --grace-period=0 <pod-name>} & Einen Pod sofort löschen, damit das ReplicaSet ihn neu erstellen kann \\
\texttt{kubectl scale replicaset nginx-replicaset --replicas=5} & Skalierung eines ReplicaSets über einen Konsolenbefehl\\
\hline
\end{tabular}
\subsubsection{YAML-Datei für die Skalierung eines ReplicaSets}
\begin{minted}[frame=lines, bgcolor=bg]{yaml}
apiVersion: apps/v1
kind: ReplicaSet
metadata:
  name: nginx-replicaset
spec:
  replicas: 5 # Skalierung auf 5 Replikate
  selector:
    matchLabels:
      app: nginx
  template:
    metadata:
      labels:
        app: nginx
    spec:
      containers:
        - name: nginx
          image: nginx:latest
          ports:
            - containerPort: 80

\end{minted}

\subsection{Fehlerbehebung bei ReplicaSets}
\begin{itemize}
    \item \textbf{Pod startet nicht:} Die Logs des Pods mit \enquote{\texttt{kubectl logs <pod-name>}} überprüfen, um mögliche Fehlerursachen zu identifizieren.
    \item \textbf{Nicht genügend Ressourcen:} Sicherstellen, dass genug Ressourcen im Cluster verfügbar sind, oder Ressourcenanforderungen des ReplicaSets anpassen
    \item \textbf{Konflikte bei Labels:} Die Label-Selektoren überprüfen, um sicherzustellen, dass keine Konflikte mit anderen ReplicaSets oder Workloads bestehen.
    \item \textbf{Unzureichende Replikate:} \enquote{\texttt{kubectl get rs}} und \enquote{\texttt{kubectl describe rs <replicaset-name>}} verwenden, um den Status und eventuelle Probleme zu überprüfen.
\end{itemize}

\subsection{Nützliche Links und Ressourcen}
Kubernetes ReplicaSet Dokumentation:\\
\url{https://kubernetes.io/docs/concepts/workloads/controllers/replicaset/}\\


\section{Jobs}
Ermöglichen die einmalige Ausführung von Aufgaben, bis sie erfolgreich abgeschlossen sind. \\

\noindent
\begin{tabular}{
|l|l|}
\hline
\textbf{Befehl} & \textbf{Beschreibung} \\
\hline
\texttt{kubectl get jobs} & Alle Jobs auflisten \\
\texttt{kubectl describe job <job-name>} & Details zu Job anzeigen \\
\texttt{kubectl create job -f <job.yaml>} & Job erstellen\\
\texttt{kubectl apply job -f <job.yaml>} & Job erstellen, oder aktualisieren\\
\texttt{kubectl delete job <job-name>} & Job löschen \\
\hline
\end{tabular}

\subsubsection{YAML-Datei für einen Job}
\begin{minted}[frame=lines, bgcolor=bg]{yaml}
apiVersion: batch/v1
kind: Job
metadata:
  name: example-job
spec:
  template:
    spec:
      containers:
        - name: example
          image: busybox
          command:
            - echo "Hello, Kubernetes!"
      restartPolicy: Never
  backoffLimit: 4

\end{minted}

\subsection{Anwendungsfälle für Jobs}
\begin{itemize}
    \item Datenverarbeitung: Ausführen von einmaligen Datenverarbeitungsaufgaben oder Batch-Jobs.
    \item Migrationen: Durchführen von Datenbank- oder Schema-Migrationen.
    \item Wartungsaufgaben: Periodische Wartungsaufgaben, die einmalig ausgeführt werden müssen.
\end{itemize}

\subsection{Best Practices für Jobs}
\begin{itemize}
    \item eine \enquote{\texttt{restartPolicy}} von \enquote{\texttt{Never}} oder \enquote{\texttt{OnFailure}} verwenden, um sicherzustellen, dass fehlgeschlagene Jobs nicht endlos neu gestartet werden.
    \item \enquote{\texttt{backoffLimit}} verwenden, um die maximale Anzahl von Wiederholungsversuchen für einen fehlgeschlagenen Job zu begrenzen.
    \item den Status von Jobs regelmäßig überwachen, um sicherzustellen, dass sie wie erwartet abgeschlossen werden.
    \item Labels und Annotations verwenden, um Jobs besser zu organisieren und zu identifizieren.
\end{itemize}

\subsection{Weitere nützliche Befehle für Jobs}
\begin{tabular}{|p{0.53\textwidth}|p{0.47\textwidth}|}
\hline
\textbf{Befehl} & \textbf{Beschreibung} \\
\hline
\texttt{kubectl get job <job-name> -o yaml} & Die vollständige YAML-Konfiguration eines Jobs anzeigen \\
\texttt{kubectl logs job/<job-name>} & Logs eines Jobs anzeigen \\
\texttt{kubectl get pods --selector=job-name=<job-name>} & Die Pods anzeigen, die zu einem bestimmten Job gehören \\
\texttt{kubectl wait --for=condition=complete job/<job-name>} & Warten, bis ein Job abgeschlossen ist \\
\texttt{kubectl patch job <job-name> -p <patch-data>} & Einen Patch auf einen Job anwenden \\
\hline
\end{tabular}

\subsection{Fehlerbehebung bei Jobs}
\begin{itemize}
    \item \textbf{Job startet nicht:} Die Logs des Jobs mit \texttt{kubectl logs job/<job-name>} überprüfen, um mögliche Fehlerursachen zu identifizieren.
    \item \textbf{Pod des Jobs bleibt hängen:} Den Status und die Logs des Pods, der dem Job zugeordnet ist überprüfen.
    \item \textbf{Job wird zu oft wiederholt:} Sicherstellen, dass die \texttt{backoffLimit} korrekt gesetzt ist, um unnötige Wiederholungen zu vermeiden.
    \item \textbf{Job läuft zu lange:} \texttt{activeDeadlineSeconds} verwenden, um die maximale Laufzeit eines Jobs zu begrenzen.
\end{itemize}

\subsection{Nützliche Links und Ressourcen}
Kubernetes Job Dokumentation:\\
\url{https://kubernetes.io/docs/concepts/workloads/controllers/job/}\\
Kubernetes Tasks für Jobs:\\
\url{https://kubernetes.io/docs/tasks/job/}\\




\section{CronJobs}
Ermöglicht die zeitgesteuerte Ausführung von Jobs basierend auf Cron-Syntax. CronJobs sind nützlich für wiederkehrende Aufgaben wie regelmäßige Backups, Batch-Verarbeitungen oder andere zeitgesteuerte Prozesse. Sie bieten eine Möglichkeit, Jobs zu bestimmten Zeiten oder in regelmäßigen Intervallen automatisch auszuführen.\\

\noindent
\begin{tabular}{|l|l|}
\hline
\textbf{Befehl} & \textbf{Beschreibung} \\
\hline
\texttt{kubectl get cronjobs} & Alle CronJobs auflisten \\
\texttt{kubectl describe cronjob <cronjob-name>} & Details zu einem CronJob anzeigen \\
\texttt{kubectl create -f cronjob.yaml} & CronJob erstellen\\
\texttt{kubectl apply -f cronjob.yaml} & CronJob erstellen oder aktualisieren\\
\texttt{kubectl delete cronjob <cronjob-name>} & Einen CronJob löschen \\
\hline
\end{tabular}
\subsubsection{Grundlagen eines CronJobs}
Ein CronJob erstellt auf Basis eines vordefinierten Zeitplans automatisch Jobs. Der Zeitplan wird in der Cron-Syntax angegeben, die aus fünf Feldern besteht, die die Minute, Stunde, Tag des Monats, Monat und Tag der Woche definieren.
\newpage
\subsubsection{YAML-Syntax für CronJobs}
\begin{minted}[frame=lines, bgcolor=bg]{yaml}
apiVersion: batch/v1beta1
kind: CronJob
metadata:
  name: beispiel-cronjob
spec:
  schedule: '0 0 * * *'
  jobTemplate:
    spec:
      template:
        spec:
          containers:
            - name: beispiel-container
              image: beispiel-image
              args:
                - /bin/sh
                - -c
                - echo "Dies ist ein Beispiel-CronJob"
          restartPolicy: OnFailure

\end{minted}

\subsection{Anwendungsfälle für CronJobs}
\begin{itemize}
    \item Regelmäßige Backups: Automatisierung von Datenbank- oder Dateisystem-Backups.
    \item Batch-Verarbeitung: Durchführung von periodischen Datenverarbeitungsaufgaben.
    \item Wartungsaufgaben: Automatisierte Wartungsprozesse wie Log-Rotation oder Cleanup-Skripte.
    \item Monitoring und Updates: Regelmäßige Überprüfungen und Aktualisierungen von Systemzuständen.
\end{itemize}

\subsection{Best Practices für CronJobs}
\begin{itemize}
    \item eine eindeutige \enquote{\texttt{schedule}} verwenden, um sicherzustellen, dass CronJobs nicht gleichzeitig ausgeführt werden.
    \item eine geeignete \enquote{\texttt{concurrencyPolicy}} setzen, um die gleichzeitige Ausführung von Jobs zu steuern (z.B. \enquote{\texttt{Forbid}} oder \enquote{\texttt{Replace}}).
    \item \enquote{\texttt{successfulJobsHistoryLimit}} und \enquote{\texttt{failedJobsHistoryLimit}} verwenden, um die Anzahl der gespeicherten erfolgreichen und fehlgeschlagenen Jobs zu begrenzen.
    \item CronJobs regelmäßig überwachen, um sicherzustellen, dass sie wie erwartet ausgeführt werden.
    \item Ressourcenanforderungen und -limits setzen, um die Stabilität des Clusters zu gewährleisten.
\end{itemize}

\subsection{Weitere nützliche Befehle für CronJobs}
\begin{tabular}{|p{0.78\textwidth}|p{0.23\textwidth}|}
\hline
\textbf{Befehl} & \textbf{Beschreibung} \\
\hline
\texttt{kubectl get cronjob <cronjob-name> -o yaml} & Konfiguration anzeigen \\
\texttt{kubectl logs job/\$(kubectl get jobs --selector=cronjob-name=<cronjob-name> -o jsonpath="{.items[0].metadata.name}")} & Logs anzeigen \\
\texttt{kubectl create job --from=cronjob/<cronjob-name> <job-name>} & Erstellen eines Jobs \\
\texttt{kubectl get jobs --selector=cronjob-name=<cronjob-name>} & Jobs auflisten \\
\texttt{kubectl patch cronjob <cronjob-name> -p <patch-data>} & Patch anwenden \\
\texttt{kubectl delete job --selector=cronjob-name=<cronjob-name>} & Alle Jobs löschen \\
\hline
\end{tabular}

\subsection{Fehlerbehebung bei CronJobs}
\begin{itemize}
    \item \textbf{Job startet nicht:} Logs des CronJobs und des erstellten Jobs mit folgendem Befehl überprüfen, um mögliche Fehlerursachen zu identifizieren:\\
    \texttt{kubectl logs job/\$(kubectl get jobs --selector=cronjob-name=<cronjob-name> -o\\jsonpath="{.items[0].metadata.name}")}
    \item \textbf{CronJob wird nicht ausgeführt:} sicherstellen, dass der Zeitplan \enquote{\texttt{schedule}} korrekt ist und keine Syntaxfehler enthält. Zeit und Datumseinstellungen des Clusters überprüfen.
    \item \textbf{Jobs bleiben hängen:} Die Pods, die von den Jobs erstellt wurden überprüfen, um zu sehen, ob sie in einem Fehlerzustand sind. \enquote{\texttt{kubectl describe pod <pod-name>}} für detaillierte Informationen verwenden.
    \item \textbf{Mehrere Instanzen eines Jobs:} \enquote{\texttt{concurrencyPolicy}} auf \enquote{\texttt{Forbid}} oder \enquote{\texttt{Replace}} setzen, um die gleichzeitige Ausführung mehrerer Instanzen eines Jobs zu verhindern.
    \item \textbf{CronJob führt Jobs zu häufig aus:}Zeitplan \enquote{\texttt{schedule}} überprüfen und korrigieren, um sicherzustellen, dass er den gewünschten Ausführungsintervall widerspiegelt.
\end{itemize}
\newpage
\subsection{Erweiterte Konfigurationsoptionen für CronJobs}
\subsubsection{Erweiterte YAML-Syntax für CronJobs}
\begin{minted}[frame=lines, bgcolor=bg, breaklines]{yaml}
apiVersion: batch/v1
kind: CronJob
metadata:
  name: erweiterter-cronjob
spec:
  schedule: "*/5 * * * *"
  concurrencyPolicy: Forbid
  successfulJobsHistoryLimit: 3
  failedJobsHistoryLimit: 1
  jobTemplate:
    spec:
      backoffLimit: 4
      template:
        spec:
          containers:
            - name: beispiel-container
              image: busybox
              args:
                - /bin/sh
                - -c
                - echo "Dies ist ein erweiterter CronJob"
          restartPolicy: OnFailure
\end{minted}

\subsection{Nützliche Links und Ressourcen}
Kubernetes CronJob Dokumentation:\\
\url{https://kubernetes.io/docs/concepts/workloads/controllers/cron-jobs/}\\
Cron Expression Generator \& Explainer:\\
\url{https://crontab.guru/}\\
