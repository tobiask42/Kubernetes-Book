\chapter{Konfiguration und Geheimnisse}
\section{ConfigMaps}
ConfigMaps werden verwendet, um Konfigurationsdaten von Anwendungen zu speichern.\\

\noindent
\begin{tabularx}{\textwidth}{|X|X|}
\hline
\textbf{Befehl} & \textbf{Beschreibung} \\
\hline
\texttt{kubectl get configmaps} & Alle ConfigMaps auflisten \\
\texttt{kubectl describe configmap <configmap-name>} & Details zu einer ConfigMap anzeigen \\
\texttt{kubectl get configmap <configmap-name>} & ConfigMap anzeigen\\
\texttt{kubectl create configmap <configmap-name>} & Neue ConfigMap erstellen \\
\texttt{kubectl delete configmap <configmap-name>} & Eine ConfigMap löschen \\
\texttt{kubectl edit configmap <configmap-name>} & Eine ConfigMap bearbeiten\\
\hline
\end{tabularx}

\subsection{Beispiele für die Erstellung von ConfigMaps}

\subsubsection{Erstellen einer ConfigMap aus einer Datei}
Eine ConfigMap kann aus einer oder mehreren Dateien erstellt werden.
\begin{minted}[frame=lines, bgcolor=bg, breaklines]{bash}
kubectl create configmap my-config --from-file=config-file.conf
\end{minted}

\subsubsection{Erstellen einer ConfigMap aus Literalwerten}
\begin{minted}[frame=lines, bgcolor=bg, breaklines]{bash}
kubectl create configmap my-config --from-literal=key1=value1 --from-literal=key2=value2
\end{minted}

\subsection{Verwendung von ConfigMaps in Pods}
ConfigMaps können in Pods als Umgebungsvariablen oder als Dateien im Dateisystem verwendet werden.

\subsubsection{Verwendung als Umgebungsvariable}
Verwendung einer ConfigMap in einem Pod als Umgebungsvariable:
\begin{minted}[frame=lines, bgcolor=bg]{yaml}
apiVersion: v1
kind: Pod
metadata:
  name: my-pod
spec:
  containers:
    - name: my-container
      image: my-image
      env:
        - name: CONFIG_KEY
          valueFrom:
            configMapKeyRef:
              name: my-config
              key: key1
\end{minted}

\subsubsection{Verwendung als Datei im Dateisystem}
Verwendung einer ConfigMap in einem Pod als Datei im Dateisystem:
\begin{minted}[frame=lines, bgcolor=bg]{yaml}
apiVersion: v1
kind: Pod
metadata:
  name: my-pod
spec:
  containers:
    - name: my-container
      image: my-image
      volumeMounts:
        - name: config-volume
          mountPath: /etc/config
  volumes:
    - name: config-volume
      configMap:
        name: my-config
\end{minted}


\subsection{Anzeigen und Bearbeiten von ConfigMaps}
ConfigMap anzeigen:
\begin{minted}[frame=lines, bgcolor=bg, breaklines]{bash}
kubectl get configmap my-config -o yaml
\end{minted}
\noindent
ConfigMap bearbeiten:
\begin{minted}[frame=lines, bgcolor=bg, breaklines]{bash}
kubectl edit configmap my-config
\end{minted}

\section{Secrets}
Secrets werden verwendet, um sensible Informationen wie Passwörter und Tokens zu speichern.\\

\noindent
\begin{tabularx}{\textwidth}{|p{0.5\textwidth}|X|}
\hline
\textbf{Befehl} & \textbf{Beschreibung} \\
\hline
\texttt{kubectl get secrets} & Alle Secrets auflisten \\
\texttt{kubectl describe secret <secret-name>} & Details zu einem Secret anzeigen \\
\texttt{kubectl create secret <type> <secret-name>} & Neues Secret erstellen \\
\texttt{kubectl delete secret <secret-name>} & Ein Secret löschen \\
\hline
\end{tabularx}

\subsection{Beispiele für die Erstellung von Secrets}

\subsubsection{Erstellen eines generischen Secrets aus Literalwerten}

\begin{minted}[frame=lines, bgcolor=bg, breaklines]{bash}
kubectl create secret generic my-secret --from-literal=username=admin --from-literal=password=secret
\end{minted}

\subsubsection{Erstellen eines Secrets aus einer Datei}

\begin{minted}[frame=lines, bgcolor=bg, breaklines]{bash}
kubectl create secret generic my-secret --from-file=./secret-file.txt
\end{minted}

\subsubsection{Erstellen eines Docker-Registry Secrets}

\begin{minted}[frame=lines, bgcolor=bg, breaklines]{bash}
kubectl create secret docker-registry my-registry-secret --docker-username=<username> --docker-password=<password> --docker-email=<email> --docker-server=<server>
\end{minted}
\subsubsection{Erstellen eines TLS Secrets}
Ein TLS Secret kann verwendet werden, um TLS-Zertifikate und -Schlüssel zu speichern:
\begin{minted}[frame=lines, bgcolor=bg, breaklines]{bash}
kubectl create secret tls my-tls-secret --cert=path/to/cert/file --key=path/to/key/file
\end{minted}

\subsection{Verwendung von Secrets in Pods}
Secrets können in Pods als Umgebungsvariablen oder als Dateien im Dateisystem verwendet werden.

\subsubsection{Verwendung als Umgebungsvariable}

\begin{minted}[frame=lines, bgcolor=bg]{yaml}
apiVersion: v1
kind: Pod
metadata:
  name: my-pod
spec:
  containers:
    - name: my-container
      image: my-image
      env:
        - name: SECRET_USERNAME
          valueFrom:
            secretKeyRef:
              name: my-secret
              key: username
        - name: SECRET_PASSWORD
          valueFrom:
            secretKeyRef:
              name: my-secret
              key: password
\end{minted}

\subsubsection{Verwendung als Datei im Dateisystem}

\begin{minted}[frame=lines, bgcolor=bg]{yaml}
apiVersion: v1
kind: Pod
metadata:
  name: my-pod
spec:
  containers:
    - name: my-container
      image: my-image
      volumeMounts:
        - name: secret-volume
          mountPath: /etc/secret
  volumes:
    - name: secret-volume
      secret:
        secretName: my-secret
\end{minted}

\subsection{Transport Layer Security (TLS)}

TLS ist ein kryptografisches Protokoll, das für die Sicherung der Kommunikation über ein Computernetzwerk, insbesondere das Internet, verwendet wird. Es stellt sicher, dass die Daten während der Übertragung zwischen Client und Server verschlüsselt und vor unbefugtem Zugriff geschützt sind. 

\subsubsection{TLS-Zertifikate}
TLS-Zertifikate sind digitale Dateien, die die Identität einer Website oder eines Servers bestätigen und einen öffentlichen Schlüssel enthalten. Sie werden von vertrauenswürdigen Zertifizierungsstellen (Certificate Authorities, CAs) ausgestellt. Ein TLS-Zertifikat enthält den öffentlichen Schlüssel des Servers, den Namen des Ausstellers, den Namen des Inhabers und die Gültigkeitsdauer des Zertifikats. 

\subsubsection{TLS-Schlüssel}
Im TLS-Protokoll gibt es zwei Hauptarten von Schlüsseln:
\begin{itemize}
    \item \textbf{Öffentlicher Schlüssel:} Dieser wird im TLS-Zertifikat bereitgestellt und dient zur Verschlüsselung von Nachrichten, die nur mit dem entsprechenden privaten Schlüssel entschlüsselt werden können.
    \item \textbf{Privater Schlüssel:} Dieser bleibt geheim und wird verwendet, um Nachrichten zu entschlüsseln, die mit dem öffentlichen Schlüssel verschlüsselt wurden, sowie um digitale Signaturen zu erzeugen.
\end{itemize}

\subsubsection{TLS-Secret}
Ein TLS-Secret ist ein spezielles Kubernetes Secret, das TLS-Zertifikate und Schlüssel enthält. Diese Secrets werden verwendet, um sichere Verbindungen zwischen Kubernetes-Komponenten oder externen Diensten zu ermöglichen. TLS-Secrets enthalten typischerweise die folgenden Daten:
\begin{itemize}
    \item \textbf{tls.crt:} Das TLS-Zertifikat.
    \item \textbf{tls.key:} Der private Schlüssel.
\end{itemize}

\subsubsection{Erstellung und Verwendung von TLS-Secrets}
TLS-Secrets können in Kubernetes mit dem folgenden Befehl erstellt werden:
\begin{minted}[frame=lines, bgcolor=bg, breaklines]{bash}
kubectl create secret tls tls-secret --cert=path/to/tls.crt --key=path/to/tls.key
\end{minted}
Sobald das TLS-Secret erstellt wurde, kann es in Pods und anderen Kubernetes-Komponenten verwendet werden, um sichere Verbindungen zu konfigurieren. Zum Beispiel kann ein Ingress-Controller das TLS-Secret verwenden, um HTTPS-Verbindungen zu terminieren.

\subsection{Anzeigen und Bearbeiten von Secrets}
Inhalt eines Secrets anzeigen:
\begin{minted}[frame=lines, bgcolor=bg]{bash}
kubectl get secret my-secret -o yaml
\end{minted}
Die Daten in einem Secret werden in einem Base64-kodiertem Format dargestellt.\\
\\
\noindent
Dekodierung der Daten:
\begin{minted}[frame=lines, bgcolor=bg]{bash}
echo 'c29tZXNlY3JldA==' | base64 --decode
\end{minted}
\noindent
Bearbeiten eines Secrets:
\begin{minted}[frame=lines, bgcolor=bg]{bash}
kubectl edit secret my-secret
\end{minted}

\subsection{Sicherheit und Best Practices}

Im Umgang mit Secrets ist es besonders wichtig, Sicherheitsaspekte zu berücksichtigen. Die folgenden Maßnahmen helfen dabei, den Schutz sensibler Daten sicherzustellen:

\begin{itemize}
  \item \textbf{Secrets verschlüsseln:} Den etcd-Datenspeicher verschlüsseln, z.\,B. mit AES-256 oder einem externen KMS.
  \item \textbf{Zugriffsrechte beschränken:} Zugriff auf Secrets per RBAC auf das notwendige Minimum reduzieren.
  \item \textbf{Namespaces isolieren:} Secrets in dedizierten Namespaces speichern, um Sichtbarkeit zu begrenzen.
  \item \textbf{Secrets rotieren:} Secrets und Zertifikate regelmäßig erneuern, um ihre Gültigkeit sicherzustellen.
  \item \textbf{Kurzlebige Tokens bevorzugen:} Statt dauerhaft gültiger Secrets besser zeitlich limitierte Zugriffstoken verwenden.
  \item \textbf{Zentrale Verwaltung nutzen:} Tools wie \texttt{cert-manager}, Vault oder CSI Drivers einsetzen, um Secrets zentral zu verwalten.
  \item \textbf{Auditlogs aktivieren:} Zugriffe auf Secrets mitprotokollieren, um unautorisierte Aktivitäten zu erkennen.
  \item \textbf{Nur notwendige Mounts konfigurieren:} Secrets nur in Pods einbinden, die sie wirklich benötigen.
  \item \textbf{Keine Protokollierung von Geheimnissen:} Sicherstellen, dass sensible Informationen nicht versehentlich in Logs geschrieben werden.
  \item \textbf{StringData verwenden:} Bei der Definition von Secrets in YAML-Dateien \texttt{stringData} anstelle von \texttt{data} verwenden, um Klartext direkt zu schreiben.
  \item \textbf{Versionskontrolle vermeiden:} Secrets niemals in Git oder andere Versionskontrollsysteme einchecken
\end{itemize}

\subsubsection{Zusätzliche Ressourcen}
Offizielle Dokumentation: \url{https://kubernetes.io/docs/concepts/security/secrets-good-practices/}

\newpage
\subsection{Verschlüsselung von etcd}
Die Verschlüsselung der etcd-Datenbank in Kubernetes ist entscheidend für die Sicherheit sensibler Daten wie Secrets und ConfigMaps. Durch die Verschlüsselung der etcd-Daten wird sichergestellt, dass selbst im Falle eines unbefugten Zugriffs auf die etcd-Datenbank die Daten nicht im Klartext lesbar sind.
\subsubsection{Konfiguration für die Verschlüsselung von etcd}
Die Verschlüsselung wird durch eine Konfigurationsdatei aktiviert, die die gewünschten Provider (z.\,B. AES-CBC oder AES-GCM) und Schlüssel angibt:

\begin{minted}[frame=lines, bgcolor=bg]{yaml}
apiVersion: apiserver.config.k8s.io/v1
kind: EncryptionConfiguration
resources:
  - resources:
      - secrets
    providers:
      - aescbc:
          keys:
            - name: key1
              secret: <base64-encoded-key>
      - identity: {}
\end{minted}
\noindent
\subsubsection{Aktivierung}

Die Konfigurationsdatei muss gespeichert und der API-Server mit dem Parameter

\texttt{--encryption-provider-config=<pfad-zur-config-datei>}  

\noindent
gestartet werden. Anschließend sollte geprüft werden, ob der API-Server erfolgreich läuft und die Verschlüsselung angewendet wird.

\subsubsection{Fehlerbehebung}
Typische Fehlerquellen bei der Verschlüsselung:

\begin{itemize}
    \item Ungültiger oder nicht erreichbarer Pfad zur Konfigurationsdatei
    \item Syntaxfehler oder fehlende Felder in der YAML-Datei
    \item Falsch kodierte oder zu kurze Base64-Schlüssel
    \item Fehler in den API-Server-Logs (mit \texttt{journalctl} oder \texttt{kubectl logs})
\end{itemize}

\subsubsection{Best Practices}
\begin{itemize}
    \item \textbf{Starke Algorithmen verwenden:} AES-256 mit GCM oder CBC-Modus nutzen.
    \item \textbf{Schlüssel regelmäßig rotieren:} Schlüsselwechsel in der Encryption-Konfigurationsdatei eintragen und Rotation durch Reencryption manuell auslösen.
    \item \textbf{Verschlüsselung testen:} Vor dem Produktiveinsatz die Konfiguration in einer Testumgebung verifizieren.
    \item \textbf{Auditierung aktivieren:} API-Zugriffe auf Secrets und etcd-Zugriffe überwachen.
    \item \textbf{Replikation absichern:} Auch etcd-Replikation über TLS verschlüsseln.
    \item \textbf{Backup-Verschlüsselung beachten:} Sicherstellen, dass Backups von etcd ebenfalls verschlüsselt sind.
\end{itemize}
\subsubsection{Zusätzliche Ressourcen}

Offizielle Dokumentation: \url{https://kubernetes.io/docs/tasks/administer-cluster/encrypt-data/}\\
Security Best Practices: \url{https://kubernetes.io/docs/concepts/security/overview/}
\newpage

\section{Kubeconfig}

Die \texttt{kubeconfig}-Datei regelt den Zugriff auf Kubernetes-Cluster. Sie speichert Konfigurationsinformationen wie Cluster-Adressen, Benutzeranmeldeinformationen und Kontexte.

\noindent
Ein \textbf{Kontext} ist eine Kombination aus einem Cluster, einem Benutzer und einem Namespace. Er definiert die Umgebung, in der \texttt{kubectl}-Befehle ausgeführt werden. Durch den Wechsel des Kontexts lässt sich einfach zwischen verschiedenen Clustern oder Rollen umschalten, ohne die Konfiguration manuell zu ändern.\\

\noindent
\begin{tabular}{|p{0.53\textwidth}|p{0.47\textwidth}|}
\hline
\textbf{Befehl} & \textbf{Beschreibung} \\
\hline
\texttt{kubectl config view} & Aktuelle Kubeconfig-Datei anzeigen \\
\texttt{kubectl config get-contexts} & Alle verfügbaren Kontexte auflisten \\
\texttt{kubectl config current-context} & Aktuell verwendeten Kontext anzeigen \\
\texttt{kubectl config use-context <context-name>} & Zu einem anderen Kontext wechseln \\
\texttt{kubectl config set-context <context-name>} & Einen neuen Kontext setzen \\
\texttt{kubectl config delete-context <context-name>} & Einen bestimmten Kontext löschen \\
\texttt{kubectl config rename-context <old-name> <new-name>} & Einen vorhandenen Kontext umbenennen \\
\texttt{kubectl config set-cluster <cluster-name> --server=<server-url>} & Einen neuen Cluster hinzufügen \\
\texttt{kubectl config set-credentials <user-name> --token=<token>} & Benutzerdaten über Token hinzufügen oder ändern \\
\texttt{kubectl config set-credentials <user-name> --username=<username> --password=<password>} & Benutzeranmeldeinformationen setzen \\
\texttt{kubectl config unset <property>} & Eine bestimmte Eigenschaft aus der Kubeconfig-Datei entfernen \\
\texttt{KUBECONFIG=<path/to/kubeconfig> kubectl ...} & Eine spezifische Kubeconfig-Datei für einen Befehl verwenden \\
\hline
\end{tabular}

\subsubsection{Kubeconfig-Datei}
\begin{minted}[frame=lines, bgcolor=bg]{yaml}
apiVersion: v1
kind: Config
clusters:
  - cluster:
      certificate-authority-data: <base64-encoded-ca-cert>
      server: https://<cluster-server>
    name: beispiel-cluster
contexts:
  - context:
      cluster: beispiel-cluster
      user: beispiel-benutzer
    name: beispiel-kontext
current-context: beispiel-kontext
users:
  - name: beispiel-benutzer
    user:
      token: <user-token>
\end{minted}
\newpage
\subsection{Verwalten von Konfigurationsdateien}
Kubectl verwendet standardmäßig die \texttt{\$HOME/.kube/config}-Datei zur Speicherung der Konfigurationsinformationen. Es können jedoch auch mehrere Konfigurationsdateien verwendet und diese mit der Umgebungsvariable \texttt{KUBECONFIG} verwaltet werden.

\subsubsection{Verwenden mehrerer Konfigurationsdateien}
\texttt{KUBECONFIG}-Umgebungsvariable für mehrere Konfigurationsdateien setzen:
\begin{minted}[frame=lines, bgcolor=bg]{bash}
export KUBECONFIG=\$HOME/.kube/config:\$HOME/.kube/config-2
\end{minted}
\noindent
Konfigurationsdateien zusammenführen:
\begin{minted}[frame=lines, bgcolor=bg, breaklines]{bash}
kubectl config view --merge --flatten > \$HOME/.kube/merged-config
export KUBECONFIG=\$HOME/.kube/merged-config
\end{minted}
\subsubsection{Hinzufügen eines neuen Clusters und Kontexts}
\begin{minted}[frame=lines, bgcolor=bg, breaklines]{bash}
# Neuen Cluster hinzufügen
kubectl config set-cluster my-cluster --server=https://my-cluster-server:6443 --certificate-authority=/path/to/ca.crt
# Benutzeranmeldeinformationen hinzufügen
kubectl config set-credentials my-user --client-certificate=/path/to/client.crt --client-key=/path/to/client.key
# Einen neuen Kontext setzen
kubectl config set-context my-context --cluster=my-cluster --namespace=my-namespace --user=my-user
# Zu dem neuen Kontext wechseln
kubectl config use-context my-context
\end{minted}
\subsubsection{Überprüfen des aktuellen Kontexts}
\begin{minted}[frame=lines, bgcolor=bg]{bash}
kubectl config current-context
\end{minted}

\subsection{Best Practices für das Verwalten von Kubernetes-Konfigurationen}
\begin{itemize}
    \item Aussagekräftige Namen für Kontexte, Cluster und Benutzer wählen.
    \item \texttt{kubeconfig}-Dateien regelmäßig sichern und sicher speichern.
    \item Verschiedene Umgebungen mit separaten Konfigurationsdateien verwalten.
    \item Konfigurationen und deren Verwendungszwecke dokumentieren.
    \item Rollen- und Berechtigungsmanagement (RBAC) zur Zugriffskontrolle nutzen.
    \item Veraltete oder ungenutzte Einträge regelmäßig entfernen.
    \item Änderungen an Konfigurationsdateien mit Versionskontrolle (z.B. Git) verwalten
    \item Sensible Informationen (Tokens, Passwörter) nicht in der Kubeconfig-Datei im Klartext speichern.
    \item Zugriff auf \texttt{.kube/config} auf Lesezugriff für den Nutzer beschränken (z.B. \texttt{chmod 600}).
    \item Automatisierte Rotationen von Zugangstokens mit ServiceAccounts bevorzugen.
\end{itemize}
\subsubsection{Zusätzliche Ressourcen}
Offizielle Dokumentation: \url{https://kubernetes.io/docs/concepts/configuration/organize-cluster-access-kubeconfig/}

\subsection{Nützliche Befehle zur Fehlerbehebung}
\begin{tabular}{|p{0.53\textwidth}|p{0.47\textwidth}|}
\hline
\textbf{Befehl} & \textbf{Beschreibung} \\
\hline
\texttt{kubectl config current-context} & Den aktuell verwendeten Kontext anzeigen \\
\texttt{kubectl config view --minify} & Die aktive Konfiguration ohne redundante Informationen anzeigen \\
\texttt{kubectl config get-clusters} & Alle definierten Cluster in der Konfigurationsdatei auflisten \\
\texttt{kubectl config get-users} & Alle definierten Benutzer in der Konfigurationsdatei auflisten \\
\texttt{kubectl config get-contexts} & Alle definierten Kontexte in der Konfigurationsdatei auflisten \\
\texttt{kubectl config delete-cluster <cluster-name>} & Einen bestimmten Cluster aus der Konfigurationsdatei löschen \\
\texttt{kubectl config delete-user <user-name>} & Einen bestimmten Benutzer aus der Konfigurationsdatei löschen \\
\texttt{kubectl config rename-context <old-name> <new-name>} & Einen Kontext umbenennen \\
\hline
\end{tabular}

\subsection{Wechseln zwischen verschiedenen Kontexten}
\begin{minted}[frame=lines, bgcolor=bg]{bash}
# Wechseln zum Entwicklungs-Kontext
kubectl config use-context dev-context

# Überprüfen des aktuellen Kontexts
kubectl config current-context

# Wechseln zum Produktions-Kontext
kubectl config use-context prod-context

# Überprüfen des aktuellen Kontexts
kubectl config current-context
\end{minted}

\subsection{Verwendung von Kontexten in Skripten}
\begin{minted}[frame=lines, bgcolor=bg]{bash}
#!/bin/bash

# Setzen des Kubernetes-Kontexts
kubectl config use-context dev-context

# Überprüfen des aktuellen Kontexts
current_context=$(kubectl config current-context)
echo "Aktueller Kontext: $current_context"

# Ausführen von Kubernetes-Befehlen im aktuellen Kontext
kubectl get pods -n my-namespace

# Wechseln zum Produktions-Kontext
kubectl config use-context prod-context

# Überprüfen des aktuellen Kontexts
current_context=$(kubectl config current-context)
echo "Aktueller Kontext: $current_context"

# Ausführen von Kubernetes-Befehlen im Produktions-Kontext
kubectl get pods -n my-namespace
\end{minted}