\chapter{Backup und Wiederherstellung}

\section{Backup-Strategien}
Es gibt verschiedene Strategien, um Backups in Kubernetes durchzuführen:
\begin{itemize}
    \item \textbf{ETCD-Backup}: Sicherung der ETCD-Datenbank, die den Zustand des gesamten Kubernetes-Clusters speichert.
    \item \textbf{Persistent Volume (PV) Backup}: Sicherung der Daten, die in Persistent Volumes gespeichert sind.
    \item \textbf{Anwendungs-Backup}: Sicherung der Anwendungsdaten und Konfigurationen.
\end{itemize}

\section{ETCD-Backup}
ETCD ist das Herzstück eines Kubernetes-Clusters. Ein Backup der ETCD-Datenbank stellt sicher, dass der Clusterzustand und die Konfigurationen wiederhergestellt werden können.\\
\phantom{.}\\
\begin{tabular}{|p{0.6\textwidth}|p{0.4\textwidth}|}
\hline
\textbf{Befehl} & \textbf{Beschreibung} \\
\hline
\texttt{ETCDCTL\_API=3 etcdctl snapshot save <snapshot.db>} & Ein ETCD-Snapshot erstellen \\
\texttt{ETCDCTL\_API=3 etcdctl snapshot restore <snapshot.db>} & Ein ETCD-Snapshot wiederherstellen \\
\texttt{ETCDCTL\_API=3 etcdctl {-}{-}write-out=table snapshot status <snapshot.db>} & Den Status eines ETCD-Snapshots überprüfen \\
\hline
\end{tabular}

\section{Persistent Volume (PV) Backup}
Daten, die in Persistent Volumes gespeichert sind, müssen regelmäßig gesichert werden. Die Methoden können je nach Storage-Provider variieren.\\
\phantom{.}\\
\begin{tabular}{|l|l|}
\hline
\textbf{Befehl} & \textbf{Beschreibung} \\
\hline
\texttt{kubectl get pv} & Alle Persistent Volumes im Cluster auflisten \\
\texttt{kubectl describe pv <pv-name>} & Details eines bestimmten Persistent Volumes anzeigen \\
\texttt{<storage-provider-specific-backup-command>} & Backup eines Persistent Volumes durchführen \\
\hline
\end{tabular}

\section{Anwendungs-Backup}
Anwendungsdaten und -konfigurationen können mit Tools wie Velero gesichert werden. Velero ist ein beliebtes Open-Source-Tool für Backup und Wiederherstellung von Kubernetes-Clustern.\\
\phantom{.}\\
\begin{tabular}{|p{0.6\textwidth}|p{0.4\textwidth}|}
\hline
\textbf{Befehl} & \textbf{Beschreibung} \\
\hline
\texttt{velero install {-}{-}provider <provider> {-}{-}bucket <bucket> {-}{-}secret-file <credentials>} & Velero im Cluster installieren \\
\texttt{velero backup create <backup-name>} & Ein Backup des gesamten Clusters erstellen \\
\texttt{velero backup describe <backup-name>} & Details eines bestimmten Backups anzeigen \\
\texttt{velero backup delete <backup-name>} & Ein Backup löschen \\
\texttt{velero restore create {-}{-}from-backup <backup-name>} & Ein Backup wiederherstellen \\
\texttt{velero restore describe <restore-name>} & Details einer Wiederherstellung anzeigen \\
\hline
\end{tabular}
\newpage
\section{Best Practices für Backup und Wiederherstellung}
\begin{itemize}
    \item \textbf{Regelmäßige Backups}: Planung und Automatisierung von regelmäßigen Backups
    \item \textbf{Backup-Überprüfung}: Regelmäßiges Testen der Backups
    \item \textbf{Offsite-Backups}: Speichern der Backups an Externen Orten
    \item \textbf{Sicherheitsmaßnahmen}: Schutz von Backups durch Verschlüsselung und Zugriffskontrollen, um Datenmissbrauch zu verhindern
    \item \textbf{Dokumentation und Schulung}: Backup- und Wiederherstellungsverfahren dokumentieren und Team regelmäßig schulen
\end{itemize}