\chapter{Sicherheit und Zugriffskontrolle}

\section{Roles und RoleBindings}
Ermöglicht die Verwaltung von Berechtigungen innerhalb von Namespaces. Roles definieren die Berechtigungen für Ressourcen innerhalb eines Namespaces, während RoleBindings diese Rollen an Benutzer, Gruppen oder ServiceAccounts binden.\\

\noindent
\begin{tabularx}{\textwidth}{|X|X|}
\hline
\textbf{Befehl} & \textbf{Beschreibung} \\
\hline
\texttt{kubectl get roles} & Alle Roles im aktuellen Namespace auflisten \\
\texttt{kubectl get rolebindings} & Alle RoleBindings im aktuellen Namespace auflisten \\
\texttt{kubectl describe role <role-name>} & Details zu einer Role anzeigen \\
\texttt{kubectl describe rolebinding <rolebinding-name>} & Details zu einer RoleBinding anzeigen \\
\texttt{kubectl create -f role.yaml} & Rolle erstellen\\
\texttt{kubectl apply -f role.yaml} & Rolle erstellen oder aktualisieren\\
\texttt{kubectl create -f rolebinding.yaml} & RoleBinding erstellen\\
\texttt{kubectl apply -f rolebinding.yaml} & RoleBinding erstellen, oder aktualisieren\\
\texttt{kubectl delete role <role-name>} & Rolle löschen \\
\texttt{kubectl delete rolebinding <rolebinding-name>} & RoleBinding löschen \\
\hline
\end{tabularx}

\subsubsection{Beispielkonfiguration für eine Role}
\begin{minted}[frame=lines, bgcolor=bg]{yaml}
apiVersion: rbac.authorization.k8s.io/v1
kind: Role
metadata:
  namespace: "default"
  name: "beispiel-role"
rules:
  - apiGroups: [""]
    resources: ["pods"]
    verbs: ["get","list","watch"]
\end{minted}


\subsubsection{Beispielkonfiguration für ein RoleBinding}
\begin{minted}[frame=lines, bgcolor=bg]{yaml}
apiVersion: rbac.authorization.k8s.io/v1
kind: RoleBinding
metadata:
  name: beispiel-rolebinding
  namespace: default
subjects:
  - kind: User
    name: example-user
    apiGroup: rbac.authorization.k8s.io
roleRef:
  kind: Role
  name: beispiel-role
  apiGroup: rbac.authorization.k8s.io
\end{minted}

\newpage
\subsection{Anwendungsfälle für Roles und RoleBindings}
\begin{itemize}
    \item Beschränken des Zugriffs auf bestimmte Ressourcen innerhalb eines Namespaces.
    \item Delegieren von Verwaltungsaufgaben an bestimmte Benutzer oder ServiceAccounts.
    \item Implementieren von Least Privilege Prinzipien, um die Sicherheit zu erhöhen.
\end{itemize}

\subsection{Best Practices für Roles und RoleBindings}
\begin{itemize}
    \item spezifische Rollen mit minimalen Berechtigungen verwenden, um das Prinzip der minimalen Rechte zu gewährleisten.
    \item alle RoleBindings und deren Zweck dokumentieren, um die Verwaltung zu erleichtern.
    \item regelmäßig die vergebenen Berechtigungen überwachen und überprüfen, um sicherzustellen, dass sie noch notwendig sind.
    \item Gruppen nutzen, um Berechtigungen effizienter zu verwalten.
\end{itemize}

\subsection{Weitere nützliche Befehle für Roles und RoleBindings}
\begin{tabular}{|p{0.53\textwidth}|p{0.47\textwidth}|}
\hline
\textbf{Befehl} & \textbf{Beschreibung} \\
\hline
\texttt{kubectl get roles -o yaml} & YAML-Konfiguration aller Roles im aktuellen Namespace anzeigen \\
\texttt{kubectl get rolebindings -o yaml} & YAML-Konfiguration aller RoleBindings im aktuellen Namespace anzeigen \\
\texttt{kubectl auth can-i <verb> <resource>} & Überprüfen, ob der aktuelle Benutzer eine bestimmte Aktion auf einer Ressource ausführen kann \\
\texttt{kubectl create role <role-name> --verb=<verb> --resource=<resource>} & Eine Role mit einem bestimmten Verb und einer Ressource erstellen \\
\texttt{kubectl create rolebinding <rolebinding-name> --role=<role-name> --user=<user-name>} & Eine RoleBinding für eine bestimmte Role und einen Benutzer erstellen \\
\hline
\end{tabular}

\subsection*{Nützliche Links und Ressourcen}
\begin{itemize}
    \item \href{https://kubernetes.io/docs/reference/access-authn-authz/rbac/}{Kubernetes RBAC Dokumentation}
    \item \href{https://kubernetes.io/docs/tasks/configure-pod-container/configure-service-account/}{Kubernetes Task: Configuring Service Accounts for Pods}
    \item \href{https://kubernetes.io/blog/2017/10/using-rbac-generally-available-18/}{Kubernetes Blog: Using RBAC for Kubernetes Authorization}
\end{itemize}

\newpage

\section{ClusterRoles und ClusterRoleBindings}
Ermöglicht die Verwaltung von Cluster-weiten Berechtigungen. ClusterRoles definieren die Berechtigungen für Ressourcen im gesamten Cluster, während ClusterRoleBindings diese Rollen an Benutzer, Gruppen oder ServiceAccounts im gesamten Cluster binden.\\

\noindent
\begin{tabular}{|p{0.5\textwidth}|p{0.5\textwidth}|}
\hline
\textbf{Befehl} & \textbf{Beschreibung} \\
\hline
\texttt{kubectl get clusterroles} & Alle ClusterRoles auflisten \\
\texttt{kubectl get clusterrolebindings} & Alle ClusterRoleBindings auflisten \\
\texttt{kubectl describe clusterrole <clusterrole-name>} & Details zu einer ClusterRole anzeigen \\
\texttt{kubectl describe clusterrolebinding <clusterrolebinding-name>} & Details zu einer ClusterRoleBinding anzeigen \\
\texttt{kubectl create -f clusterrole.yaml} & ClusterRole erstellen\\
\texttt{kubectl apply -f clusterrole.yaml} & ClusterRole erstellen, oder aktualisieren\\
\texttt{kubectl create -f clusterrolebinding.yaml} & ClusterRoleBinding erstellen\\
\texttt{kubectl apply -f clusterrolebinding.yaml} & ClusterRoleBinding erstellen oder aktualisieren\\
\texttt{kubectl delete clusterrole <clusterrole-name>} & Eine ClusterRole löschen \\
\texttt{kubectl delete clusterrolebinding <clusterrolebinding-name>} & Eine ClusterRoleBinding löschen \\
\hline
\end{tabular}

\subsubsection{Beispielkonfiguration für eine ClusterRole}
\begin{minted}[frame=lines, bgcolor=bg]{yaml}
apiVersion: rbac.authorization.k8s.io/v1
kind: ClusterRole
metadata:
  name: beispiel-clusterrole
rules:
  - apiGroups: [""]
    resources: ["pods"]
    verbs: ["get", "list", "watch"]
\end{minted}

\subsubsection{Beispielkonfiguration für ein ClusterRoleBinding}
\begin{minted}[frame=lines, bgcolor=bg]{yaml}
apiVersion: rbac.authorization.k8s.io/v1
kind: ClusterRoleBinding
metadata:
  name: beispiel-clusterrolebinding
subjects:
  - kind: User
    name: example-user
    apiGroup: rbac.authorization.k8s.io
roleRef:
  kind: ClusterRole
  name: beispiel-clusterrole
  apiGroup: rbac.authorization.k8s.io
\end{minted}

\subsection{Anwendungsfälle für ClusterRoles und ClusterRoleBindings}
\begin{itemize}
    \item Zuweisen von administrativen Berechtigungen für Cluster-weite Operationen.
    \item Delegieren von Berechtigungen für spezifische Cluster-weite Aufgaben an bestimmte Benutzer oder ServiceAccounts.
    \item Implementieren von Sicherheitsrichtlinien auf Cluster-Ebene.
\end{itemize}

\subsection{Best Practices für ClusterRoles und ClusterRoleBindings}
\begin{itemize}
    \item spezifische Rollen mit minimalen Berechtigungen verwenden, um das Prinzip der minimalen Rechte zu gewährleisten.
    \item alle ClusterRoleBindings und deren Zweck dokumentieren, um die Verwaltung zu erleichtern.
    \item regelmäßig die vergebenen Berechtigungen überwachen und überprüfen, um sicherzustellen, dass sie noch notwendig sind.
    \item Gruppen nutzen, um Berechtigungen effizienter zu verwalten.
    \item Namenskonventionen für ClusterRoles und ClusterRoleBindings verwenden, um ihre Zwecke klar zu kommunizieren.
\end{itemize}

\subsection{Weitere nützliche Befehle für ClusterRoles und ClusterRoleBindings}
\begin{tabular}{
|p{0.53\textwidth}|p{0.47\textwidth}|}
\hline
\textbf{Befehl} & \textbf{Beschreibung} \\
\hline
\texttt{kubectl get clusterroles -o yaml} & YAML-Konfiguration aller ClusterRoles anzeigen \\
\texttt{kubectl get clusterrolebindings -o yaml} & YAML-Konfiguration aller ClusterRoleBindings anzeigen \\
\texttt{kubectl auth can-i <verb> <resource> --all-namespaces} & Überprüfen, ob der aktuelle Benutzer eine bestimmte Aktion auf einer Ressource in allen Namespaces ausführen kann \\
\texttt{kubectl create clusterrole <clusterrole-name> --verb=<verb> --resource=<resource>} & Eine ClusterRole mit einem bestimmten Verb und einer Ressource erstellen \\
\texttt{kubectl create clusterrolebinding <clusterrolebinding-name> --clusterrole=<clusterrole-name> --user=<user-name>} & Eine ClusterRoleBinding für eine bestimmte ClusterRole und einen Benutzer erstellen \\
\hline
\end{tabular}

\subsection*{Nützliche Links und Ressourcen}
\begin{itemize}
    \item \href{https://kubernetes.io/docs/reference/access-authn-authz/rbac/}{Kubernetes RBAC Dokumentation}
    \item \href{https://kubernetes.io/docs/tasks/configure-pod-container/configure-service-account/}{Kubernetes Task: Configuring Service Accounts for Pods}
    \item \href{https://kubernetes.io/blog/2017/10/using-rbac-generally-available-18/}{Kubernetes Blog: Using RBAC for Kubernetes Authorization}
\end{itemize}

\section{ServiceAccounts}
ServiceAccounts werden verwendet, um Pods mit bestimmten Berechtigungen auszustatten. Sie ermöglichen Pods den Zugriff auf den Kubernetes API-Server und andere Ressourcen unter Verwendung von Role-Based Access Control (RBAC).\\

\noindent
\begin{tabular}{
|p{0.5\textwidth}|p{0.5\textwidth}|}
\hline
\textbf{Befehl} & \textbf{Beschreibung} \\
\hline
\texttt{kubectl get serviceaccounts} & Alle ServiceAccounts im aktuellen Namespace auflisten \\
\texttt{kubectl describe serviceaccount <serviceaccount-name>} & Details zu einem ServiceAccount anzeigen \\
\texttt{kubectl create serviceaccount <serviceaccount-name>} & Einen neuen ServiceAccount erstellen \\
\texttt{kubectl apply -f serviceaccount.yaml} & Serviceaccount aus Datei erstellen, oder aktualisieren\\
\texttt{kubectl delete serviceaccount <serviceaccount-name>} & Einen ServiceAccount löschen \\
\hline
\end{tabular}

\subsubsection{Beispielkonfiguration für einen ServiceAccount}
\begin{minted}[frame=lines, bgcolor=bg]{yaml}
apiVersion: v1
kind: ServiceAccount
metadata:
  name: beispiel-serviceaccount
  namespace: default
\end{minted}

\subsection{Anwendungsfälle für ServiceAccounts}
\begin{itemize}
    \item Ermöglichen von Pods, sich gegenüber dem Kubernetes API-Server zu authentifizieren.
    \item Zuweisen spezifischer Berechtigungen zu Pods mittels RBAC.
    \item Isolieren von Pods und deren Berechtigungen, um das Prinzip der minimalen Rechte zu gewährleisten.
    \item Implementieren von Sicherheitsrichtlinien, um den Zugriff von Pods auf Cluster-Ressourcen zu steuern.
\end{itemize}

\subsection{Best Practices für ServiceAccounts}
\begin{itemize}
    \item Spezifische ServiceAccounts für verschiedene Anwendungen oder Pods erstellen, anstatt den Standard-ServiceAccount zu verwenden.
    \item RoleBindings oder ClusterRoleBindings verwenden, um die Berechtigungen für ServiceAccounts zu steuern.
    \item Regelmäßig die vergebenen Berechtigungen überprüfen, um sicherzustellen, dass sie noch notwendig sind.
    \item Alle ServiceAccounts und deren Zweck dokumentieren, um die Verwaltung zu erleichtern.
    \item Namenskonventionen für ServiceAccounts nutzen, um ihre Zwecke klar zu kommunizieren.
\end{itemize}

\subsection{Weitere nützliche Befehle für ServiceAccounts}
\begin{tabular}{|p{0.56\textwidth}|p{0.44\textwidth}|}
\hline
\textbf{Befehl} & \textbf{Beschreibung} \\
\hline
\texttt{kubectl get serviceaccounts -o yaml} & YAML-Konfiguration aller ServiceAccounts im aktuellen Namespace anzeigen \\
\texttt{kubectl patch serviceaccount <serviceaccount-name> -p <patch-data>} & Einen Patch auf einen ServiceAccount anwenden \\
\texttt{kubectl get secret <secret-name> -o yaml} & Das Secret eines ServiceAccounts anzeigen \\
\texttt{kubectl describe secret <secret-name>} & Details zu einem Secret eines ServiceAccounts anzeigen \\
\texttt{kubectl create rolebinding <rolebinding-name> --role=<role-name> --serviceaccount=<namespace>:<serviceaccount-name>} & Eine RoleBinding für einen bestimmten ServiceAccount erstellen \\
\texttt{kubectl create clusterrolebinding <clusterrolebinding-name> --clusterrole=<clusterrole-name> --serviceaccount=<namespace>:<serviceaccount-name>} & Eine ClusterRoleBinding für einen bestimmten ServiceAccount erstellen \\
\hline
\end{tabular}
\newpage
\subsection{Komplettbeispiel}
\begin{minted}[frame=lines, bgcolor=bg]{yaml}
apiVersion: v1
kind: Namespace
metadata:
  name: rbac-demo
\end{minted}
\begin{minted}[frame=lines, bgcolor=bg]{yaml}
apiVersion: v1
kind: ServiceAccount
metadata:
  name: demo-sa
  namespace: rbac-demo
\end{minted}
\begin{minted}[frame=lines, bgcolor=bg]{yaml}
apiVersion: rbac.authorization.k8s.io/v1
kind: Role
metadata:
  namespace: rbac-demo
  name: pod-reader
rules:
  - apiGroups: [""]
    resources: ["pods"]
    verbs: ["get", "list", "watch"]
\end{minted}
\begin{minted}[frame=lines, bgcolor=bg]{yaml}
apiVersion: rbac.authorization.k8s.io/v1
kind: RoleBinding
metadata:
  name: read-pods
  namespace: rbac-demo
subjects:
  - kind: ServiceAccount
    name: demo-sa
    namespace: rbac-demo
roleRef:
  kind: Role
  name: pod-reader
  apiGroup: rbac.authorization.k8s.io
\end{minted}
\begin{minted}[frame=lines, bgcolor=bg]{yaml}
apiVersion: v1
kind: Pod
metadata:
  name: pod-uses-sa
  namespace: rbac-demo
spec:
  serviceAccountName: demo-sa
  containers:
    - name: curl-container
      image: curlimages/curl:latest
      command: ["sleep", "3600"]
\end{minted}
\subsection*{Nützliche Links und Ressourcen}
\begin{itemize}
    \item \href{https://kubernetes.io/docs/tasks/configure-pod-container/configure-service-account/}{Kubernetes Task: Configuring Service Accounts for Pods}
    \item \href{https://kubernetes.io/docs/reference/access-authn-authz/service-accounts-admin/}{Kubernetes ServiceAccounts Administration}
    \item \href{https://kubernetes.io/blog/2016/08/security-best-practices-kubernetes-deployment/}{Kubernetes Blog: Security Best Practices for Kubernetes Deployment}
\end{itemize}


\newpage

\section{Network Policies}
Network Policies ermöglichen die Kontrolle des Datenverkehrs zwischen Pods und anderen Netzwerkendpunkten auf Basis von Label-Selektoren und Regeln. Sie bieten eine Möglichkeit, den ein- und ausgehenden Datenverkehr zu regulieren, um die Sicherheit und Isolation im Cluster zu erhöhen. \\

\noindent
\begin{tabular}{
|p{0.45\textwidth}|p{0.55\textwidth}|}
\hline
\textbf{Befehl} & \textbf{Beschreibung} \\
\hline
\texttt{kubectl get networkpolicies} & Alle Network Policies im aktuellen Namespace auflisten \\
\texttt{kubectl describe networkpolicy <policy-name>} & Details zu einer bestimmten Network Policy anzeigen \\
\texttt{kubectl create -f <networkpolicy.yaml>} & Eine neue Network Policy anhand einer YAML-Datei erstellen \\
\texttt{kubectl apply -f <networkpolicy.yaml>} & Eine Network Policy erstellen, oder aktualisieren\\
\texttt{kubectl delete networkpolicy <policy-name>} & Eine Network Policy löschen \\
\texttt{kubectl edit networkpolicy <policy-name>} & Eine Network Policy im Editor bearbeiten \\
\texttt{kubectl get networkpolicy <policy-name> -o yaml} & Eine Network Policy im YAML-Format anzeigen \\
\hline
\end{tabular}

\subsubsection{Beispielkonfiguration für eine Network Policy}
\begin{minted}[frame=lines, bgcolor=bg]{yaml}
apiVersion: networking.k8s.io/v1
kind: NetworkPolicy
metadata:
  name: beispiel-networkpolicy
  namespace: default
spec:
  podSelector:
    matchLabels:
      role: db
  policyTypes:
    - Ingress
    - Egress
  ingress:
    - from:
        - podSelector:
            matchLabels:
              role: frontend
      ports:
        - protocol: TCP
          port: 3306
  egress:
    - to:
        - podSelector:
            matchLabels:
              role: backend
      ports:
        - protocol: TCP
          port: 3306

\end{minted}
\newpage
\subsection{Anwendungsfälle für Network Policies}
\begin{itemize}
    \item Isolieren von Pods, um sicherzustellen, dass nur autorisierter Datenverkehr zugelassen wird.
    \item Schutz sensibler Datenbanken, indem der Zugriff nur auf spezifische Pods beschränkt wird.
    \item Implementieren von Sicherheitsrichtlinien, um unautorisierten Datenverkehr zu verhindern.
    \item Erzwingen von Kommunikationsmustern zwischen verschiedenen Anwendungskomponenten.
\end{itemize}

\subsection{Best Practices für Network Policies}
\begin{itemize}
    \item Beginne mit restriktiven Policies und erweitere schrittweise die erlaubten Verbindungen.
    \item Nutze Label-Selektoren, um granularere Kontrollen über den Datenverkehr zu erlangen.
    \item Überprüfe und teste regelmäßig die Network Policies, um sicherzustellen, dass sie die gewünschten Sicherheitsanforderungen erfüllen.
    \item Dokumentiere alle Network Policies und deren Zweck, um die Verwaltung zu erleichtern.
    \item Nutze Namespaces, um Network Policies auf bestimmte Bereiche des Clusters zu beschränken.
\end{itemize}

\subsection{Weitere nützliche Befehle für Network Policies}
\begin{tabular}{|p{0.53\textwidth}|p{0.47\textwidth}|}
\hline
\textbf{Befehl} & \textbf{Beschreibung} \\
\hline
\texttt{kubectl get networkpolicies -o yaml} & YAML-Konfiguration aller Network Policies im aktuellen Namespace anzeigen \\
\texttt{kubectl patch networkpolicy <policy-name> -p <patch-data>} & Einen Patch auf eine Network Policy anwenden \\
\texttt{kubectl label pod <pod-name> <label-key>=<label-value>} & Ein Label zu einem Pod hinzufügen, um die Network Policy anzuwenden \\
\texttt{kubectl annotate networkpolicy <policy-name> <annotation-key>=<annotation-value>} & Eine Annotation zu einer Network Policy hinzufügen \\
\texttt{kubectl get pods --selector=<label-selector>} & Alle Pods mit einem bestimmten Label-Selektor auflisten \\
\hline
\end{tabular}

\subsection{Nützliche Links und Ressourcen}
\begin{itemize}
    \item \href{https://kubernetes.io/docs/concepts/services-networking/network-policies/}{Kubernetes Network Policies Dokumentation}
    \item \href{https://kubernetes.io/docs/tasks/administer-cluster/declare-network-policy/}{Kubernetes Task: Declare Network Policy}
    \item \href{https://github.com/ahmetb/kubernetes-network-policy-recipes}{Kubernetes Network Policy Recipes auf GitHub}
\end{itemize}
