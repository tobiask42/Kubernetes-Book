\chapter{Erweiterungen und Anpassungen}

\section{Plugins für Kubectl}
Kubectl-Plugins erweitern die Funktionalität des Standard-Kubectl-Befehlszeilenwerkzeugs. Sie ermöglichen benutzerdefinierte Befehle und Automatisierungen, die speziell auf die Bedürfnisse eines Clusters oder eines Teams zugeschnitten sind. \\

\noindent
\begin{tabular}{|p{0.5\textwidth}|p{0.5\textwidth}|}
\hline
\textbf{Befehl} & \textbf{Beschreibung} \\
\hline
\texttt{kubectl plugin list} & Kubectl-Plugins auflisten \\
\texttt{kubectl plugin <plugin-name>} & Spezifischen Plugin-Befehl ausführen \\
\texttt{kubectl krew install <plugin-name>} & Plugin installieren \\
\texttt{kubectl krew uninstall <plugin-name>} & Plugin deinstallieren \\
\texttt{kubectl krew list} & Installierte Plugins anzeigen \\
\texttt{kubectl krew update} & Krew-Plugin-Datenbank aktualisieren \\
\texttt{kubectl krew search <plugin-name>} & Nach Plugins suchen \\
\hline
\end{tabular}

\subsection{Krew: Der Plugin-Manager für Kubectl}
Krew ist ein Plugin-Manager für Kubectl, der die Installation und Verwaltung von Plugins vereinfacht. Mit Krew können Plugins aus einer zentralen Plugin-Datenbank durchsucht, installiert, aktualisiert und deinstalliert werden.

\subsubsection{Installation von Krew}
Um Krew zu installieren, kann das folgende Skript verwendet werden:
\begin{minted}[frame=lines, bgcolor=bg, breaklines]{bash}
(
  set -x; cd "$(mktemp -d)" &&
  OS="$(uname | tr '[:upper:]' '[:lower:]')" &&
  ARCH="$(uname -m |

 sed -e 's/x86_64/amd64/' -e 's/arm.*$/arm/')" &&
  KREW="krew-${OS}_${ARCH}" &&
  curl -fsSLO "https://github.com/kubernetes-sigs/krew/releases/latest/ download/${KREW}.tar.gz" &&
  tar zxvf "${KREW}.tar.gz" &&
  ./"${KREW}" install krew
)
export PATH="${KREW_ROOT:-$HOME/.krew}/bin:$PATH"
\end{minted}

\subsubsection{Verwalten von Plugins mit Krew}
Mit Krew können verschiedene Verwaltungsaufgaben für Plugins durchgeführt werden:
\begin{minted}[frame=lines, bgcolor=bg]{bash}
# Nach einem Plugin suchen
kubectl krew search <plugin-name>
# Ein Plugin installieren
kubectl krew install <plugin-name>
# Ein Plugin deinstallieren
kubectl krew uninstall <plugin-name>
# Alle installierten Plugins auflisten
kubectl krew list
# Die Krew-Plugin-Datenbank aktualisieren
kubectl krew update
\end{minted}

\subsection{Beliebte Kubectl-Plugins}
\begin{itemize}
    \item \texttt{kubectl neat}: Entfernt unnötige Felder aus der Ausgabe von Kubectl-Befehlen
    \item \texttt{kubectl ctx}: Ermöglicht das schnelle Wechseln zwischen Kubernetes-Kontexten
    \item \texttt{kubectl ns}: Ermöglicht das schnelle Wechseln zwischen Namespaces
    \item \texttt{kubectl tree}: Zeigt eine hierarchische Ansicht von Kubernetes-Ressourcen
    \item \texttt{kubectl view-secret}: Zeigt den Inhalt von Secrets in Klartext an
\end{itemize}

\subsection{Erstellen eigener Kubectl-Plugins}
Eigene Kubectl-Plugins können erstellt werden, um spezifische Anforderungen zu erfüllen. Ein Kubectl-Plugin ist im Grunde ein ausführbares Skript oder Programm, das im Pfad verfügbar ist und mit \texttt{kubectl-<plugin-name>} benannt ist.

Hier ist ein Beispiel für ein einfaches Kubectl-Plugin, das in Bash geschrieben ist und eine Liste der Nodes im Cluster ausgibt:
\begin{minted}[frame=lines, bgcolor=bg, breaklines]{bash}
#!/bin/bash

# Datei als kubectl-nodes.sh speichern und ausführbar machen
kubectl get nodes -o custom-columns=NAME:.metadata.name,STATUS:.status.conditions [-1].type
\end{minted}

Um dieses Skript als Kubectl-Plugin zu verwenden, sind folgende Schritte notwendig:

\begin{enumerate}
    \item Das Skript als \texttt{kubectl-nodes} speichern.
    \item Die Datei ausführbar machen:\\
    \texttt{chmod +x kubectl-nodes}
    \item Das Skript in ein Verzeichnis legen, das im \texttt{PATH} enthalten ist, oder den \texttt{PATH} entsprechend anpassen:\\
    \texttt{export PATH=\$PATH:/pfad/zum/skript}
\end{enumerate}

Nach diesen Schritten kann das Plugin mit folgendem Befehl ausgeführt werden:\\
\texttt{kubectl nodes}

\subsection{Sicherheitsaspekte bei der Nutzung von Plugins}
Beim Einsatz von Kubectl-Plugins sind einige Sicherheitsaspekte zu beachten:

\begin{itemize}
    \item Nur Plugins aus vertrauenswürdigen Quellen installieren, um das Risiko von Malware oder bösartigem Code zu minimieren.
    \item Die Skripte und ausführbaren Dateien der Plugins regelmäßig überprüfen, um sicherzustellen, dass sie keine unerwünschten Änderungen enthalten.
    \item Die Berechtigungen der Plugin-Skripte und -Dateien streng kontrollieren, um unbefugten Zugriff zu verhindern.
    \item Sicherheitsupdates für installierte Plugins beachten und diese regelmäßig aktualisieren.
\end{itemize}

\section{Custom Resource Definitions (CRDs)}
\label{sec:crd}
Custom Resource Definitions (CRDs) ermöglichen die Erweiterung von Kubernetes um benutzerdefinierte Ressourcen. Sie sind eine Möglichkeit, um eigene API-Objekte zu definieren und zu verwalten, die über die Standard-Ressourcen von Kubernetes hinausgehen.\\

\noindent
\begin{tabular}{
|p{0.45\textwidth}|p{0.55\textwidth}|}
\hline
\textbf{Befehl} & \textbf{Beschreibung} \\
\hline
\texttt{kubectl get crds} & Alle Custom Resource Definitions auflisten \\
\texttt{kubectl describe crd <crd-name>} & Details zu einer Custom Resource Definition anzeigen \\
\texttt{kubectl create -f <crd.yaml>} & Custom Resource Definition erstellen\\
\texttt{kubectl apply -f <crd.yaml>} & Custom Resource Definition erstellen, oder updaten\\
\texttt{kubectl delete crd <crd-name>} & Eine Custom Resource Definition löschen \\
\hline
\end{tabular}

\subsubsection{YAML-Datei für eine Custom Resource Definition}
\begin{minted}[frame=lines, bgcolor=bg]{yaml}
---
apiVersion: apiextensions.k8s.io/v1
kind: CustomResourceDefinition
metadata:
  name: beispiel-crd.example.com
spec:
  group: example.com
  versions:
    - name: v1
      served: true
      storage: true
      schema:
        openAPIV3Schema:
          type: object
          properties:
            spec:
              type: object
              properties:
                foo:
                  type: string
  scope: Namespaced
  names:
    plural: beispiel-crds
    singular: beispiel-crd
    kind: BeispielCRD
    shortNames:
      - bcrd
\end{minted}

\subsection{Anwendungsfälle für Custom Resource Definitions}
\begin{itemize}
    \item Definieren und Verwalten benutzerdefinierter Ressourcen, die spezifische Anforderungen erfüllen.
    \item Erweiterung der Kubernetes-API um projekt- oder domänenspezifische Funktionalitäten.
    \item Automatisierung und Verwaltung komplexer Anwendungslogik und -konfigurationen.
    \item Integration von Drittanbieter-Tools und -Services in Kubernetes.
\end{itemize}

\subsection{Best Practices für Custom Resource Definitions}
\begin{itemize}
    \item Vermeide Namenskonflikte durch die Verwendung eindeutiger Gruppen- und Ressourcennamen.
    \item Verwende Versionskontrolle für CRDs, um die Abwärtskompatibilität sicherzustellen.
    \item Dokumentiere die Struktur und das Schema der CRDs klar und umfassend.
    \item Überprüfe und teste CRDs gründlich, bevor sie in der Produktion verwendet werden.
    \item Verwende Validierungs- und Konvertierungs-Webhook-Server, um die Konsistenz und Integrität der benutzerdefinierten Ressourcen sicherzustellen.
\end{itemize}

\subsection{Weitere nützliche Befehle für Custom Resource Definitions}
\begin{tabular}{|p{0.53\textwidth}|p{0.47\textwidth}|}
\hline
\textbf{Befehl} & \textbf{Beschreibung} \\
\hline
\texttt{kubectl get crd <crd-name> -o yaml} & Konfiguration einer CRD anzeigen \\
\texttt{kubectl edit crd <crd-name>} & Custom Resource Definition bearbeiten \\
\texttt{kubectl patch crd <crd-name> -p <patch-data>} & Patch anwenden \\
\texttt{kubectl get <crd-name>} & Ressourcen für eine bestimmte CRD auflisten \\
\texttt{kubectl describe <crd-name> <resource-name>} & Details  anzeigen \\
\texttt{kubectl delete <crd-name> <resource-name>} & Ressource löschen \\
\hline
\end{tabular}

\subsection*{Nützliche Links und Ressourcen}
Kubernetes Custom Resource Definitions Dokumentation:\\
{\normalsize\ttfamily\href{https://kubernetes.io/docs/tasks/extend-kubernetes/custom-resources/custom-resource-definitions/}{https://kubernetes.io/docs/tasks/extend-kubernetes/custom-resources/custom-resource-definitions/}}
Kubernetes Konzepte: Custom Resources:\\
\url{https://kubernetes.io/docs/concepts/extend-kubernetes/api-extension/custom-resources/}\\
Beispiel-Controller für Custom Resources auf GitHub:\\
\url{https://github.com/kubernetes/sample-controller}
\subsection{Erstellen und Verwalten einer benutzerdefinierten Ressource}
Beispiel, bei dem eine benutzerdefinierte Ressource namens \texttt{BeispielResource} erstellt und verwaltet wird.

\subsubsection{YAML-Datei für die benutzerdefinierte Ressource}
\begin{minted}[frame=lines, bgcolor=bg]{yaml}
apiVersion: example.com/v1
kind: BeispielResource
metadata:
  name: beispiel-resource-instance
  namespace: default
spec:
  foo: bar
\end{minted}


\subsubsection{Erstellen einer benutzerdefinierten Ressource aus einer YAML-Datei}
\begin{minted}[frame=lines, bgcolor=bg]{bash}
kubectl apply -f beispiel-resource-instance.yaml
\end{minted}

\subsubsection{Auflisten und Anzeigen der benutzerdefinierten Ressourcen}
\begin{minted}[frame=lines, bgcolor=bg]{bash}
kubectl get beispielresources
kubectl describe beispielresource beispiel-resource-instance
\end{minted}

\subsubsection{Löschen einer benutzerdefinierten Ressource}
\begin{minted}[frame=lines, bgcolor=bg]{bash}
kubectl delete beispielresource beispiel-resource-instance
\end{minted}
\newpage
\subsection{Controller für Custom Resources}
Um die Funktionalität von Custom Resources vollständig zu nutzen, kann ein benutzerdefinierter Controller entwickelt werden, der die Lebenszyklen der benutzerdefinierten Ressourcen verwaltet. Ein Controller überwacht die API-Server nach Änderungen an der benutzerdefinierten Ressource und führt entsprechende Aktionen aus.

\subsubsection{Beispiel-Controller für \texttt{BeispielResource}}
Ein Beispiel-Controller kann in Go geschrieben werden und verwendet das \href{https://github.com/kubernetes/client-go}{client-go} Paket von Kubernetes. Der Grundlegende Code für einen solchen Controller würde wie folgt aussehen:
\begin{minted}[frame=lines, bgcolor=bg, linenos]{go}
// Reduziertes Beispiel eines Go-basierten Controllers mit client-go
package main

import (
    "context"
    "fmt"
    "time"

    "k8s.io/client-go/kubernetes"
    "k8s.io/client-go/rest"
    "k8s.io/client-go/tools/cache"
)

func main() {
    config, err := rest.InClusterConfig()
    if err != nil {
        panic(err.Error())
    }
    clientset, err := kubernetes.NewForConfig(config)
    if err != nil {
        panic(err.Error())
    }

    exampleResourceInformer := cache.NewSharedInformer(
        // Informer Konfiguration für BeispielResource
    )

    stopCh := make(chan struct{})
    defer close(stopCh)

    go exampleResourceInformer.Run(stopCh)

    for {
        select {
        case <-stopCh:
            fmt.Println("Stopping controller")
            return
        default:
            fmt.Println("Controller running")
            time.Sleep(10 * time.Second)
        }
    }
}
\end{minted}
\newpage
\noindent
Der komplette Code für den funktionstüchtigen Controller sieht wie folgt aus:
\begin{minted}[frame=lines, bgcolor=bg, linenos]{go}
package main

import (
    "fmt"
    "os"
    "time"

    "k8s.io/client-go/informers"
    "k8s.io/client-go/kubernetes"
    "k8s.io/client-go/tools/cache"
    "k8s.io/client-go/tools/clientcmd"
)

func main() {
    // Laden der Kubernetes-Konfiguration (innerhalb oder außerhalb des Clusters)
    config, err := clientcmd.BuildConfigFromFlags("", clientcmd.RecommendedHomeFile)
    if err != nil {
        fmt.Fprintf(os.Stderr, "Error building kubeconfig: %s\n", err.Error())
        os.Exit(1)
    }
    // Erstellen des Clientsets
    clientset, err := kubernetes.NewForConfig(config)
    if err != nil {
        fmt.Fprintf(os.Stderr, "Error creating clientset: %s\n", err.Error())
        os.Exit(1)
    }
    // Erstellen eines Factory für SharedInformers
    factory := informers.NewSharedInformerFactory(clientset, 10*time.Minute)
    // Informer für Pods erstellen
    podInformer := factory.Core().V1().Pods().Informer()
    stopCh := make(chan struct{})
    defer close(stopCh)
    // Event-Handler für den Pod-Informer hinzufügen
    podInformer.AddEventHandler(cache.ResourceEventHandlerFuncs{
        AddFunc: func(obj interface{}) {
            fmt.Println("Pod added:", obj)
        },
        UpdateFunc: func(oldObj, newObj interface{}) {
            fmt.Println("Pod updated:", newObj)
        },
        DeleteFunc: func(obj interface{}) {
            fmt.Println("Pod deleted:", obj)
        },
    })
    // Informer starten
    go podInformer.Run(stopCh)
    // Warten auf Synchronisierung der Caches
    if !cache.WaitForCacheSync(stopCh, podInformer.HasSynced) {
        fmt.Fprintf(os.Stderr, "Error waiting for cache sync\n")
        os.Exit(1)
    }
    fmt.Println("Controller running")
    select {}
}

\end{minted}

\newpage
\section{Helm}
Helm ist ein Paketmanager für Kubernetes, der das Verwalten von Kubernetes-Anwendungen vereinfacht. Mit Helm können Anwendungen als Charts (Pakete mit vorkonfigurierten Kubernetes-Ressourcen) installiert, aktualisiert und verwaltet werden.\\

\noindent
\begin{tabular}{
|p{0.45\textwidth}|p{0.55\textwidth}|}
\hline
\textbf{Befehl} & \textbf{Beschreibung} \\
\hline
\texttt{helm list} & Alle installierten Helm-Releases auflisten \\
\texttt{helm install <release-name> <chart>} & Ein neues Helm-Chart installieren \\
\texttt{helm upgrade <release-name> <chart>} & Ein Helm-Release aktualisieren \\
\texttt{helm delete <release-name>} & Ein Helm-Release löschen \\
\texttt{helm repo add <repo-name> <repo-url>} & Ein Helm-Repository hinzufügen \\
\texttt{helm repo update} & Alle Helm-Repositories aktualisieren \\
\hline
\end{tabular}
\subsubsection{Helm Installieren}
\texttt{sudo snap install helm {-}{-}classic}

\subsubsection{Helm-Chart-Struktur}
Ein Helm-Chart ist ein Paket, das alle notwendigen Konfigurationsdateien enthält, um eine Kubernetes-Anwendung zu deployen. Die typische Struktur eines Helm-Charts sieht wie folgt aus:

\begin{minted}[frame=lines, bgcolor=bg]{text}
 mychart
  ├── Chart.yaml        # Metadaten über das Chart
  ├── values.yaml       # Standardwerte für die Konfiguration
  ├── charts            # Abhängige Charts
  └── templates         # Kubernetes-Manifestdateien
      └── _helpers.tpl  # Hilfsfunktionen
\end{minted}

\subsubsection{Erstellen eines neuen Helm-Charts}
Um ein neues Helm-Chart zu erstellen, verwendet man den folgenden Befehl:
\begin{minted}[frame=lines, bgcolor=bg]{bash}
helm create mychart
\end{minted}

\subsubsection{Anpassen von Werten in einem Helm-Chart}
Die Werte in einem Helm-Chart können durch Erstellen einer eigenen \texttt{values.yaml}-Datei überschrieben werden. Diese Datei enthält die benutzerdefinierten Konfigurationen für das Chart.
\begin{minted}[frame=lines, bgcolor=bg]{yaml}
replicaCount: 2
image:
  repository: myrepo/myimage
  tag: 1.0.0
  pullPolicy: IfNotPresent
service:
  type: LoadBalancer
  port: 80
\end{minted}

\subsubsection{Installieren eines Helm-Charts mit benutzerdefinierten Werten}
Um ein Helm-Chart mit benutzerdefinierten Werten zu installieren, verwendet man den folgenden Befehl:
\begin{minted}[frame=lines, bgcolor=bg]{bash}
helm install <release-name> <chart> -f values.yaml
\end{minted}
\newpage
\subsection{Anwendungsfälle für Helm}
\begin{itemize}
    \item Vereinfachung des Deployments und der Verwaltung von Kubernetes-Anwendungen.
    \item Bereitstellung und Aktualisierung von Anwendungen mit minimalem Aufwand.
    \item Verwaltung von Abhängigkeiten zwischen verschiedenen Kubernetes-Ressourcen.
    \item Wiederverwendbarkeit und gemeinsame Nutzung von Konfigurationen und Best Practices.
    \item Erleichterung der Zusammenarbeit in Teams durch standardisierte Charts.
\end{itemize}

\subsection{Best Practices für die Nutzung von Helm}
\begin{itemize}
    \item Helm-Charts sorgfältig versionieren, um die Rückverfolgbarkeit und Reproduzierbarkeit zu gewährleisten.
    \item \texttt{values.yaml}-Dateien verwenden, um Konfigurationen zu abstrahieren und flexibel zu gestalten.
    \item Helm-Repositorys nutzen um Charts zu teilen und wiederzuverwenden.
    \item Charts gründlich in einer Staging-Umgebung testen, bevor sie in der Produktion verwendet werden.
    \item Die Verwendung und Konfiguration von Charts klar und umfassend dokumentieren.
\end{itemize}

\subsection{Weitere nützliche Befehle für Helm}
\begin{tabular}{
|p{0.53\textwidth}|p{0.47\textwidth}|}
\hline
\textbf{Befehl} & \textbf{Beschreibung} \\
\hline
\texttt{helm search hub <keyword>} & Nach Charts suchen \\
\texttt{helm show values <chart>} & Standardwerte anzeigen \\
\texttt{helm show chart <chart>} & Metadaten anzeigen \\
\texttt{helm show readme <chart>} & README-Datei anzeigen \\
\texttt{helm dependency update} & Abhängigkeiten eines Charts aktualisieren \\
\texttt{helm rollback <release-name> <revision>} & Revision rückgängig machen \\
\texttt{helm history <release-name>} & Versionshistorie anzeigen \\
\hline
\end{tabular}

\subsection*{Nützliche Links und Ressourcen}
Offizielle Helm-Dokumentation:\\
\url{https://helm.sh/docs/}\\
Artifact Hub für Helm-Charts:\\
\url{https://artifacthub.io/}\\
Helm Blog für Neuigkeiten und Updates:\\
\url{https://helm.sh/blog/}\\
Helm GitHub Repository:\\
\url{https://github.com/helm/helm}\\
Best Practices für Helm-Charts:\\
\url{https://helm.sh/docs/chart_best_practices/}

\newpage
\subsection{Erstellen und Verwalten eines Helm-Charts}

\subsubsection{Erstellen eines neuen Helm-Charts}
\begin{minted}[frame=lines, bgcolor=bg]{bash}
helm create mychart
\end{minted}

\subsubsection{Installieren des Helm-Charts}
\begin{minted}[frame=lines, bgcolor=bg]{bash}
helm install my-release mychart
\end{minted}

\subsubsection{Überprüfen des Helm-Releases}
\begin{minted}[frame=lines, bgcolor=bg]{bash}
helm list
helm status my-release
\end{minted}

\subsubsection{Anpassen der \texttt{values.yaml}-Datei}
\begin{minted}[frame=lines, bgcolor=bg]{yaml}
replicaCount: 3
image:
  repository: nginx
  tag: 1.19.2
  pullPolicy: IfNotPresent
service:
  type: ClusterIP
  port: 80
\end{minted}


\subsubsection{Aktualisieren des Helm-Releases}
Nach bearbeitung der \texttt{values.yaml}-Datei:
\begin{minted}[frame=lines, bgcolor=bg]{bash}
helm upgrade my-release mychart -f values.yaml
\end{minted}

\subsubsection{Löschen des Helm-Releases}
\begin{minted}[frame=lines, bgcolor=bg]{bash}
helm delete my-release
\end{minted}

\newpage

\section{Operators}
Operators sind Kubernetes-Controller, die speziell dafür entwickelt wurden, komplexe Anwendungen und Zustandsmaschinen zu verwalten. Sie nutzen benutzerdefinierte Ressourcen (Custom Resources), um Anwendungen und deren Betriebsaufgaben zu automatisieren. Operators erweitern die Fähigkeiten von Kubernetes, indem sie Anwendungslogik und Betriebswissen in den Cluster einbringen.\\

\noindent
\begin{tabular}{|p{0.6\textwidth}|p{0.4\textwidth}|}
\hline
\textbf{Befehl} & \textbf{Beschreibung} \\
\hline
\texttt{kubectl get operators} & Operators mit CRD auflisten \\
\texttt{kubectl get crds} & CRDs auflisten \\
\texttt{kubectl get <custom-resource>} & Instanzen einer Ressource auflisten \\
\texttt{kubectl describe <custom-resource> <resource-name>} & Details anzeigen \\
\texttt{kubectl create -f <custom-resource.yaml>} & Ressource erstellen \\
\texttt{kubectl delete <custom-resource> <resource-name>} & Ressource löschen \\
\texttt{kubectl edit <custom-resource> <resource-name>} & Ressource bearbeiten \\
\texttt{kubectl get <custom-resource> <resource-name> -o yaml} & Ressource anzeigen \\
\texttt{kubectl logs <operator-pod-name>} & Logs eines Operator-Pods anzeigen \\
\texttt{kubectl get pods -l name=<operator-name>} & Pods eines Operators auflisten \\
\texttt{kubectl describe pod <operator-pod-name>} & Details zu Operator-Pod anzeigen \\
\texttt{kubectl apply -f <operator-deployment.yaml>} & Operator bereitstellen \\
\texttt{kubectl delete -f <operator-deployment.yaml>} & Operator entfernen \\
\hline
\end{tabular}

\subsection{Anwendungsfälle für Operators}
\begin{itemize}
    \item Automatisierung des Lebenszyklusmanagements von stateful Anwendungen wie Datenbanken und Message Queues.
    \item Implementierung von komplexen Geschäftslogiken und Betriebsabläufen als Kubernetes-Ressourcen.
    \item Vereinfachung der Verwaltung und Skalierung von Anwendungen durch die Automatisierung wiederkehrender Aufgaben.
    \item Sicherstellung der Einhaltung von Best Practices und Unternehmensrichtlinien durch codierte Betriebslogik.
\end{itemize}

\subsection{Best Practices für die Entwicklung und Nutzung von Operators}
\begin{itemize}
    \item Mit klar definierten Anwendungsanforderungen beginnen und identifizieren, welche Betriebsaufgaben automatisiert werden können.
    \item Geeignete Frameworks wie \texttt{Operator SDK} oder \texttt{Kubebuilder} zur Entwicklung von Operators verwenden.
    \item Umfassende Tests für den Operator implementieren, um sicherzustellen, dass er unter verschiedenen Szenarien korrekt funktioniert.
    \item Die benutzerdefinierten Ressourcen und deren Nutzung klar und umfassend dokumentieren.
    \item Die Aktivitäten des Operators überwachen und loggen, um Probleme frühzeitig zu erkennen und zu beheben.
    \item Die Versionierung des Operators sorgfältig verwalten, um Kompatibilitätsprobleme zu vermeiden.
\end{itemize}

\subsection{Frameworks und Tools zur Entwicklung von Operators}
Operator SDK: Framework zur schnellen Entwicklung von Operators in Go, Ansible oder Helm.\\
\url{https://sdk.operatorframework.io/}\\
Kubebuilder: Framework zur Entwicklung von Kubernetes APIs und Controllern mit CRDs.\\
\url{https://book.kubebuilder.io/}\\
Metacontroller: Controller, der es ermöglicht, benutzerdefinierte Controller durch Konfiguration und Scripts zu erstellen.\\
\url{https://metacontroller.github.io/metacontroller/}

\subsection{Erstellen eines einfachen Operators mit Operator SDK}
Anleitung für neueste Version\\
\url{https://sdk.operatorframework.io/docs/installation/#install-from-github-release}

\subsubsection{Installation des Operator SDK}
\begin{minted}[frame=lines, bgcolor=bg, breaklines]{bash}
curl -Lo operator-sdk https://github.com/operator-framework/operator-sdk/releases/ download/v1.9.0/operator-sdk_linux_amd64
chmod +x operator-sdk
mv operator-sdk /usr/local/bin/
\end{minted}


\subsubsection{Erstellen eines neuen Operator-Projekts}
\begin{minted}[frame=lines, bgcolor=bg, breaklines]{bash}
operator-sdk init --domain=example.com --repo=github.com/example/my-operator
\end{minted}

\subsubsection{Definieren der benutzerdefinierten Ressource (CRD)}
\begin{minted}[frame=lines, bgcolor=bg, linenos]{bash}
# Eine neue API und einen Controller erstellen
operator-sdk create api --group cache --version v1alpha1 --kind Memcached --resource --controller
\end{minted}
Die Datei \texttt{api/v1alpha1/memcached\_types.go} bearbeiten, um das Schema der benutzerdefinierten Ressource zu definieren:
\begin{minted}[frame=lines, bgcolor=bg, linenos]{go}
// MemcachedSpec defines the desired state of Memcached
type MemcachedSpec struct {
    Size int32 `json:"size"`
}

// MemcachedStatus defines the observed state of Memcached
type MemcachedStatus struct {
    Nodes []string `json:"nodes"`
}
\end{minted}

\subsubsection{Implementieren des Controllers}
Die Datei \texttt{controllers/memcached\_controller.go} bearbeiten, um die Geschäftslogik des Controllers zu implementieren:
\begin{minted}[frame=lines, bgcolor=bg, linenos, breaklines]{go}
// Reconcile-Methode implementieren, um gewünschte und tatsächliche Zustände abzugleichen
func (r *MemcachedReconciler) Reconcile(ctx context.Context, req ctrl.Request) (ctrl.Result, error) {
    // Logik zum Abgleich des Zustands
}
\end{minted}

\subsubsection{Erstellen und Anwenden der CRD}
\begin{minted}[frame=lines, bgcolor=bg]{bash}
# Generiere die Manifeste
make manifests

# Wende die CRD auf den Cluster an
kubectl apply -f config/crd/bases
\end{minted}

\subsubsection{Bereitstellen des Operators}
\begin{minted}[frame=lines, bgcolor=bg]{bash}
# Operator-Image erstellen und in ein Container-Registry pushen
make docker-build docker-push IMG=example/my-operator:v0.1.0

# Deployment Manifeste anwenden
make deploy IMG=example/my-operator:v0.1.0
\end{minted}

\subsubsection{Verwalten der benutzerdefinierten Ressource}
Instanz der benutzerdefinierten Ressource erstellen, um den Operator zu testen:
\begin{minted}[frame=lines, bgcolor=bg]{yaml}
apiVersion: cache.example.com/v1alpha1
kind: Memcached
metadata:
  name: example-memcached
spec:
  size: 3
\end{minted}
\begin{minted}[frame=lines, bgcolor=bg]{bash}
# Benutzerdefinierte Ressource erstellen
kubectl apply -f config/samples/cache_v1alpha1_memcached.yaml

# Status des Operators und der Ressourcen überprüfen
kubectl get memcached
kubectl describe memcached example-memcached
\end{minted}