\clearpage
\thispagestyle{empty}
\vspace*{\fill}

\begin{center}
  {\normalfont\huge\bfseries Vorwort}
\end{center}
\vspace{1em}
\addcontentsline{toc}{chapter}{Vorwort}

{
\setlength{\parindent}{0pt}
\setlength{\parskip}{0.5em}

\begin{adjustwidth}{2.5cm}{2.5cm}

Dieses Dokument entstand im Rahmen eines selbstgewählten Lernprojekts innerhalb eines 14-monatigen Praktikums. Ein Teilbereich dieses Praktikums beinhaltete die Einarbeitung in Kubernetes und verwandte Infrastrukturthemen. Zur Strukturierung des Lernprozesses begann ich einen Coursera-Kurs und hielt parallel eigene Notizen fest, aus denen eine erste Rohfassung dieses Dokuments hervorging.

Die Inhalte basieren in erster Linie auf der offiziellen Dokumentation, wurden jedoch durch Fachartikel, selbst entwickelte Beispiele und eigene praktische Erfahrungen ergänzt.

Nach Abschluss des Praktikums wurde das Material überarbeitet, erweitert und zu einer eigenständigen technischen Übersicht ausgebaut. Neben der inhaltlichen Verfeinerung wurden auch Layout und Titelgestaltung überarbeitet. Die aktuelle Fassung versteht sich als praxisnahes Nachschlagewerk für zentrale Kubernetes-Konzepte, typische Anwendungsfälle und bewährte Vorgehensweisen.

Die Kapitel behandeln Aufgabenfelder wie Konfiguration, Skalierung, Speicherverwaltung, Sicherheit und CI/CD. Die Inhalte sind mit Beispielen, praktischen Hinweisen und realitätsnahen Erklärungen versehen, um sowohl das Verständnis der Konzepte als auch deren Anwendung in der Praxis zu fördern.

\end{adjustwidth}
}

\vfill
\clearpage
