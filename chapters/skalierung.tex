\chapter{Skalierung und Autoscaling}

\section{Autoscalers}
Kubernetes unterstützt verschiedene Mechanismen zur automatischen Skalierung von Workloads und Infrastruktur. Diese Skalierungsarten lassen sich in drei Hauptkategorien unterteilen:
\begin{itemize}
    \item \textbf{Horizontal Pod Autoscaler (HPA):} Skalierung der Anzahl von Pods.
    \item \textbf{Vertical Pod Autoscaler (VPA):} Anpassung der Ressourcenlimits von Pods.
    \item \textbf{Cluster Autoscaler:} Skalierung der Anzahl der Nodes im Cluster.
\end{itemize}


\subsection{Horizontal Pod Autoscalers (HPA)}
Der HPA skaliert die Anzahl von Pods in einem Deployment, StatefulSet oder ReplicaSet basierend auf Metriken wie CPU-Auslastung oder benutzerdefinierten Indikatoren. \\

\noindent
\begin{tabular}{|p{0.53\textwidth}|p{0.47\textwidth}|}
\hline
\texttt{kubectl get hpa} & Liste aller HPAs \\
\texttt{kubectl autoscale deployment <deployment-name>} & Deployment automatisch skalieren \\
\texttt{kubectl describe hpa <hpa-name>} & Details eines HPAs anzeigen \\
\texttt{kubectl delete hpa <hpa-name>} & HPA löschen \\
\texttt{kubectl edit hpa <hpa-name>} & HPA bearbeiten \\
\texttt{kubectl scale --replicas=<number> deployment <deployment-name>} & Manuelle Skalierung der Pods \\
\hline
\end{tabular}
\phantom{.}\\
\noindent
\subsubsection{Beispielkonfiguration eines Horizontal Pod Autoscalers}
\begin{minted}[frame=lines, bgcolor=bg]{yaml}
apiVersion: autoscaling/v1
kind: HorizontalPodAutoscaler
metadata:
  name: beispiel-hpa
spec:
  scaleTargetRef:
    apiVersion: apps/v1
    kind: Deployment
    name: beispiel-deployment
  minReplicas: 1
  maxReplicas: 10
  targetCPUUtilizationPercentage: 50
\end{minted}

\newpage
\subsection{Vertical Pod Autoscalers (VPA)}
Automatische Anpassung der Ressourcenanforderungen (CPU und Speicher) von Pods anhand der tatsächlichen Nutzung über die Zeit. \\

\noindent
\begin{tabular}{|p{0.53\textwidth}|p{0.47\textwidth}|}
\hline
\texttt{kubectl get vpa} & Liste aller VPAs \\
\texttt{kubectl describe vpa <vpa-name>} & Details eines VPAs anzeigen \\
\texttt{kubectl delete vpa <vpa-name>} & VPA löschen \\
\texttt{kubectl apply -f <vpa-config.yaml>} & VPA aus YAML erstellen \\
\texttt{kubectl edit vpa <vpa-name>} & VPA bearbeiten \\
\texttt{kubectl get vpa -o yaml} & YAML-Ausgabe aller VPAs \\
\texttt{kubectl patch vpa <vpa-name> --patch '<json-patch>'} & VPA patchen \\
\hline
\end{tabular}
\subsubsection{Beispielkonfiguration eines Vertical Pod Autoscalers:}
\begin{minted}[frame=lines, bgcolor=bg]{yaml}
apiVersion: autoscaling/v1
kind: HorizontalPodAutoscaler
metadata:
  name: beispiel-hpa
spec:
  scaleTargetRef:
    apiVersion: apps/v1
    kind: Deployment
    name: beispiel-deployment
  minReplicas: 1
  maxReplicas: 10
  targetCPUUtilizationPercentage: 50
\end{minted}



\subsection{Cluster Autoscalers}
Der Cluster Autoscaler erhöht oder reduziert automatisch die Anzahl der Nodes im Cluster, je nach Ressourcenbedarf der Pods. Nicht schedulbare Pods oder unterausgelastete Nodes sind typische Trigger. \\

\noindent
\begin{tabular}{|p{0.53\textwidth}|p{0.47\textwidth}|}
\hline
\texttt{kubectl get cluster-autoscaler} & Liste aller Cluster Autoscaler \\
\texttt{kubectl describe cluster-autoscaler <autoscaler-name>} & Details anzeigen \\
\texttt{kubectl delete cluster-autoscaler <autoscaler-name>} & Löschen eines Autoscalers \\
\texttt{kubectl apply -f <autoscaler-config.yaml>} & Autoscaler konfigurieren \\
\texttt{kubectl edit cluster-autoscaler <autoscaler-name>} & Autoscaler bearbeiten \\
\texttt{kubectl get cluster-autoscaler -o yaml} & YAML-Ausgabe \\
\texttt{kubectl patch cluster-autoscaler <autoscaler-name> --patch '<json-patch>'} & Patch anwenden \\
\hline
\end{tabular}
\subsubsection{Beispielkonfiguration eines Cluster Autoscalers:}
\begin{minted}[frame=lines, bgcolor=bg]{yaml}
apiVersion: autoscaling.k8s.io/v1
kind: ClusterAutoscaler
metadata:
  name: beispiel-cluster-autoscaler
spec:
  scaleDown:
    enabled: true
    unneededTime: 10m
    utilizationThreshold: 0.5
\end{minted}

\section{Manuelle Skalierung}
Die manuelle Skalierung erlaubt es, die Anzahl der Replikate eines Deployments, StatefulSets, ReplicaSets oder ReplicationControllers unabhängig von automatischen Mechanismen anzupassen.\\

\noindent
\begin{tabular}{|p{0.53\textwidth}|p{0.47\textwidth}|}
\hline
\textbf{Befehl} & \textbf{Beschreibung} \\
\hline
\texttt{kubectl scale deployment <deployment-name> --replicas=<number>} & Replikate eines Deployments setzen \\
\texttt{kubectl scale statefulset <statefulset-name> --replicas=<number>} & Replikate eines StatefulSets setzen \\
\texttt{kubectl scale rc <replication-controller-name> --replicas=<number>} & Replikate eines ReplicationControllers setzen \\
\texttt{kubectl scale replicaset <replicaset-name> --replicas=<number>} & Replikate eines ReplicaSets setzen \\
\texttt{kubectl get deployment <deployment-name> -o=jsonpath='{.spec.replicas}'} & Aktuelle Replikate eines Deployments anzeigen \\
\texttt{kubectl get statefulset <statefulset-name> -o=jsonpath='{.spec.replicas}'} & Aktuelle Replikate eines StatefulSets anzeigen \\
\texttt{kubectl get rc <replication-controller-name> -o=jsonpath='{.spec.replicas}'} & Aktuelle Replikate eines ReplicationControllers anzeigen \\
\texttt{kubectl get replicaset <replicaset-name> -o=jsonpath='{.spec.replicas}'} & Aktuelle Replikate eines ReplicaSets anzeigen  \\
\hline
\end{tabular}
\phantom{.}\\
Eine umfangreiche Erklärung zu jsonpath befindet sich bei \ref{subsec:use-json-path}

\begin{quote}
\textbf{Achtung:} Die manuelle Skalierung überschreibt temporär Einstellungen eines \texttt{HorizontalPodAutoscaler}. Bei aktiver automatischer Skalierung kann sich die manuell gesetzte Replikazahl daher wieder ändern.
\end{quote}

\subsection{Best Practices für die Skalierung}
\begin{itemize}
  \item \textbf{Ressourcenanforderungen definieren:} \texttt{resources.requests} und \texttt{resources.limits} für alle Pods angeben, um fundierte Entscheidungen durch Autoscaler zu ermöglichen.

  \item \textbf{Metriken überwachen:} CPU- und Speicherauslastung regelmäßig prüfen und gegebenenfalls benutzerdefinierte Metriken einbinden (z.\,B. via Prometheus Adapter).

  \item \textbf{Initiale Replikazahl festlegen:} Eine Ausgangsgröße wählen, die eine schnelle Verfügbarkeit sicherstellt, ohne erst bei Lastanstieg skalieren zu müssen.

  \item \textbf{HPA und VPA nicht kombinieren:} HPA und VPA nicht gleichzeitig auf dieselben Deployments anwenden, da sich ihre Wirkungen überschneiden können.

  \item \textbf{Cluster-Autoscaler begrenzen:} Minimal- und Maximalanzahl an Nodes setzen, um unkontrolliertes Skalieren oder Ressourcenmangel zu vermeiden.

  \item \textbf{Cooldown-Zeiten konfigurieren:} Eine sinnvolle Downscale-Stabilisierung (z.\,B. mit \texttt{--horizontal-pod-autoscaler-downscale-stabilization}) einstellen, um häufiges Hin- und Herschalten zu verhindern.

  \item \textbf{Skalierungsverhalten testen:} Änderungen an Skalierungsregeln in einer Testumgebung prüfen, bevor sie produktiv eingesetzt werden.
\end{itemize}

\subsubsection*{Weitere Ressourcen}
HPA:  
\url{https://kubernetes.io/docs/tasks/run-application/horizontal-pod-autoscale/}\\
Autoscaling Workloads: \url{https://kubernetes.io/docs/concepts/workloads/autoscaling/}
