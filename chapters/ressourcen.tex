\chapter{Ressourcenverwaltung}

\section{Resource Quotas und LimitRanges}
Ermöglicht es, Ressourcenlimits und -quoten für Namespaces festzulegen. Resource Quotas begrenzen die Gesamtmenge an Ressourcen, die von einem Namespace verbraucht werden können, während LimitRanges Mindest- und Höchstwerte für die Ressourcen festlegen, die einzelne Container oder Pods verwenden können.\\

\noindent
\begin{tabular}{
|p{0.5\textwidth}|p{0.5\textwidth}|}
\hline
\textbf{Befehl} & \textbf{Beschreibung} \\
\hline
\texttt{kubectl get resourcequotas} & Alle ResourceQuotas im aktuellen Namespace auflisten \\
\texttt{kubectl describe resourcequota <resourcequota-name>} & Details zu einer ResourceQuota anzeigen \\
\texttt{kubectl delete resourcequota <resourcequota-name>} & Eine ResourceQuota löschen \\
\texttt{kubectl get limitranges} & Alle LimitRanges im aktuellen Namespace auflisten \\
\texttt{kubectl create -f <resourcequota.yaml>} & eine Resource Quota erstellen\\
\texttt{kubectl create -f <limitrange.yaml>} & Eine LimitRange erstellen\\
\texttt{kubectl apply -f <resourcequota.yaml>} & eine Resource Quota erstellen oder updaten\\
\texttt{kubectl apply -f <limitrage.yaml>} & eine LimitRange erstellen oder updaten\\
\texttt{kubectl describe limitrange <limitrange-name>} & Details zu einer LimitRange anzeigen \\
\texttt{kubectl delete limitrange <limitrange-name>} & Eine LimitRange löschen \\
\hline
\end{tabular}

\subsubsection{YAML-Datei für eine ResourceQuota}
\input{minted/tex/YAML-resource-quota.tex}

\newpage
\subsubsection{YAML-Datei für eine LimitRange}
\begin{minted}[frame=lines, bgcolor=bg]{yaml}
apiVersion: v1
kind: LimitRange
metadata:
  name: beispiel-limitrange
  namespace: default
spec:
  limits:
    - max:
        cpu: "2"
        memory: 4Gi
      min:
        cpu: 200m
        memory: 6Mi
      default:
        cpu: "1"
        memory: 2Gi
      defaultRequest:
        cpu: 500m
        memory: 1Gi
      type: Container

\end{minted}

\subsection{Anwendungsfälle für Resource Quotas und LimitRanges}
\begin{itemize}
    \item Sicherstellen, dass ein Namespace nicht mehr Ressourcen als vorgesehen verbraucht.
    \item Verhindern, dass einzelne Pods oder Container zu viele Ressourcen beanspruchen.
    \item Ermöglichen einer fairen Ressourcenverteilung zwischen verschiedenen Teams oder Projekten.
    \item Schutz des Clusters vor Ressourcenengpässen durch übermäßige Nutzung.
\end{itemize}

\subsection{Best Practices für Resource Quotas und LimitRanges}
\begin{itemize}
    \item Setze realistische Quoten und Limits basierend auf der tatsächlichen Nutzung und den Anforderungen der Anwendungen.
    \item Überwache regelmäßig die Ressourcennutzung, um sicherzustellen, dass die Quoten und Limits angemessen sind.
    \item Kommuniziere die Quoten und Limits klar an alle Teams, um Verständnis und Akzeptanz zu fördern.
    \item Nutze Namespaces, um verschiedene Quoten und Limits für unterschiedliche Teams oder Projekte festzulegen.
    \item Passe die Quoten und Limits bei Änderungen der Anforderungen oder der Clusterkapazität an.
\end{itemize}

\subsection{Weitere nützliche Befehle für Resource Quotas und LimitRanges}
\begin{tabular}{|p{0.53\textwidth}|p{0.47\textwidth}|}
\hline
\textbf{Befehl} & \textbf{Beschreibung} \\
\hline
\texttt{kubectl get resourcequotas -o yaml} & YAML-Konfiguration aller ResourceQuotas im aktuellen Namespace anzeigen \\
\texttt{kubectl get limitranges -o yaml} & YAML-Konfiguration aller LimitRanges im aktuellen Namespace anzeigen \\
\texttt{kubectl patch resourcequota <resourcequota-name> -p <patch-data>} & Einen Patch auf eine ResourceQuota anwenden \\
\texttt{kubectl patch limitrange <limitrange-name> -p <patch-data>} & Einen Patch auf eine LimitRange anwenden \\
\texttt{kubectl edit resourcequota <resourcequota-name>} & Eine ResourceQuota im Editor bearbeiten \\
\texttt{kubectl edit limitrange <limitrange-name>} & Eine LimitRange im Editor bearbeiten \\
\texttt{kubectl get pods --namespace=<namespace> --field-selector=status.phase=Failed} & Alle fehlgeschlagenen Pods in einem Namespace auflisten, um Ressourcenlecks zu identifizieren \\
\hline
\end{tabular}

\subsection*{Nützliche Links und Ressourcen}
\begin{itemize}
    \item \href{https://kubernetes.io/docs/concepts/policy/resource-quotas/}{Kubernetes Resource Quotas Dokumentation}
    \item \href{https://kubernetes.io/docs/tasks/administer-cluster/quota-api-object/}{Kubernetes Task: Administering Resource Quotas}
    \item \href{https://kubernetes.io/docs/concepts/policy/limit-range/}{Kubernetes Limit Range Dokumentation}
    \item \href{https://kubernetes.io/docs/tasks/administer-cluster/manage-resources/memory-default-namespace/}{Kubernetes Task: Managing Compute Resources for Containers}
\end{itemize}



\section{PodDisruptionBudgets (PDB)}
PodDisruptionBudgets (PDB) werden verwendet, um die Anzahl der gleichzeitigen Pod-Ausfälle zu begrenzen. Sie helfen dabei, die Verfügbarkeit von Anwendungen zu gewährleisten, indem sie sicherstellen, dass eine Mindestanzahl von Pods immer verfügbar bleibt, selbst während Wartungsarbeiten oder Upgrades.\\

\noindent
\begin{tabular}{
|p{0.4\textwidth}|p{0.6\textwidth}|}
\hline
\textbf{Befehl} & \textbf{Beschreibung} \\
\hline
\texttt{kubectl get pdb} & Alle PodDisruptionBudgets im aktuellen Namespace auflisten \\
\texttt{kubectl describe pdb <pdb-name>} & Details zu einem PodDisruptionBudget anzeigen \\
\texttt{kubectl create -f <pdb.yaml>} & PodDisruptionBudget erstellen\\
\texttt{kubectl apply -f <pdb.yaml>} & PodDisruptionBudget erstellen oder updaten\\
\texttt{kubectl delete pdb <pdb-name>} & Ein PodDisruptionBudget löschen \\
\hline
\end{tabular}

\subsubsection{Beispielkonfiguration für ein PodDisruptionBudget}
\begin{minted}[frame=lines, bgcolor=bg]{yaml}
apiVersion: policy/v1
kind: PodDisruptionBudget
metadata:
  name: beispiel-pdb
  namespace: default
spec:
  minAvailable: 2
  selector:
    matchLabels:
      app: beispiel-app
\end{minted}

\newpage

\subsection{Anwendungsfälle für PodDisruptionBudgets}
\begin{itemize}
    \item Sicherstellen, dass eine Mindestanzahl von Pods während geplanter Wartungsarbeiten verfügbar bleibt.
    \item Schutz kritischer Anwendungen vor zu vielen gleichzeitigen Pod-Ausfällen.
    \item Verbesserung der Anwendungsverfügbarkeit während automatisierter Upgrades und Rollouts.
    \item Unterstützung bei der Einhaltung von Service Level Agreements (SLAs) durch Gewährleistung der Verfügbarkeit.
\end{itemize}

\subsection{Best Practices für PodDisruptionBudgets}
\begin{itemize}
    \item Setze realistische Werte für \texttt{minAvailable} oder \texttt{maxUnavailable}, um die Verfügbarkeit von Anwendungen zu gewährleisten.
    \item Überprüfe regelmäßig die Konfiguration der PDBs, um sicherzustellen, dass sie den aktuellen Anforderungen entsprechen.
    \item Kombiniere PDBs mit anderen Hochverfügbarkeitsstrategien wie ReplicaSets und Deployments.
    \item Dokumentiere alle PDBs und deren Zweck, um die Verwaltung zu erleichtern.
    \item Teste die Auswirkungen von PDBs in einer Staging-Umgebung, bevor sie in der Produktion verwendet werden.
\end{itemize}

\subsection{Weitere nützliche Befehle für PodDisruptionBudgets}
\begin{tabular}{|p{0.53\textwidth}|p{0.47\textwidth}|}
\hline
\textbf{Befehl} & \textbf{Beschreibung} \\
\hline
\texttt{kubectl get pdb -o yaml} & YAML-Konfiguration aller PodDisruptionBudgets im aktuellen Namespace anzeigen \\
\texttt{kubectl edit pdb <pdb-name>} & Ein PodDisruptionBudget im Editor bearbeiten \\
\texttt{kubectl patch pdb <pdb-name> -p <patch-data>} & Einen Patch auf ein PodDisruptionBudget anwenden \\
\texttt{kubectl get pods --selector=<label-selector>} & Alle Pods auflisten, die von einem bestimmten PDB betroffen sind \\
\hline
\end{tabular}

\subsection{Nützliche Links und Ressourcen}
\begin{itemize}
    \item \href{https://kubernetes.io/docs/concepts/workloads/pods/disruptions/}{Kubernetes PodDisruptionBudgets Dokumentation}
    \item \href{https://kubernetes.io/docs/tasks/run-application/configure-pdb/}{Kubernetes Task: Configure PodDisruptionBudget}
\end{itemize}